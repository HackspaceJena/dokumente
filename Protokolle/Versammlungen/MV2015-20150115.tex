\documentclass[ngerman]{scrartcl}
\usepackage[utf8]{inputenc}
\usepackage{lmodern}
\usepackage{courier}
\usepackage[scaled=.92]{helvet}
\usepackage[T1]{fontenc}
\usepackage[ngerman]{babel}

\usepackage{minutes}
\usepackage{eurosym}
\usepackage{tabularx}

\begin{document}
\begin{Protokoll}{Protokoll der außerordentlichen Mitgliederversammlung des Hackspace Jena e.\,V.}
\protokollant{Uwe Lippmann}
\teilnehmer{10 Vereinsmitglieder}
\moderation{Tim Schumacher}
\sitzungsdatum{15.\,Januar~2015}
\sitzungsbeginn{20:15 Uhr}
\sitzungsende{20:45 Uhr}
\sitzungsort{Krautspace, Krautgasse~26, 07743~Jena}

\protokollKopf{}

\topic{Begrüßung}
Tim Schumacher und Martin Neß begrüßen die Anwesenden.
Es befinden sich 10 ordentliche Mitglieder im Raum.
Martin Neß stellt fest, dass die Versammlung beschlussfähig ist, da laut Satzung zur Erfüllung der 23\%-Regelung mindestens 7~Mitglieder anwesend sein müssen.

\topic{Wahl des Versammlungsleiters \& Protokollführers}

Tim erklärt sich bereit Versammlungsleiter zu sein und wird einstimmig durch Handzeichen gewählt (1 Enthaltung).

Uwe wird als Protokollführer vorgeschlagen und wird einstimmig durch Handzeichen gewählt (1 Enthaltung).

\topic{Abstimmung der Tagesordnung}
Die Tagesordnung wird einstimmig durch offene Wahl mit Handzeichen angenommen.

\topic{Feststellung ordentliche Einladung}
Martin berichtet, dass die Einladung für die Versammlung mehr als 14~Tage im Voraus -- am 4.\,1.\,2015 
gegen 23:06~Uhr -- an alle Mitglieder per E-Mail sowie als Kopie an die öffentliche Mailingliste
versendet wurde.

\topic{Wahl des Schriftführers}

Es wird kurz darüber diskutiert, ob eine Stichwahl zwischen den beiden Kandidaten mit den meisten Stimmen  
(Jan und Katja) oder ob ein zweiter Wahlgang mit allen Kandidaten stattfindet.
Felix Kästner erklärt, nicht mehr kandidieren zu wollen.
Die Wahl findet mithin zwischen Jan Huwald und Katja Hagenbring statt.

Der Versammlungsleiter fragt, ob eines der Mitglieder eine geheime Wahl wünscht.
Die Mitglieder verneinen. Es findet öffentliche Wahl durch Handzeichen statt.

Ergebnis der Abstimmung zur Wahl des Schriftführers:
Katja erhält 7 Stimmen, Jan erhält 3 Stimmen.

\beschluss{Schriftführer}{Katja Hagenbring nimmt die Wahl an und ist
damit als Schriftführer gewählt.}

\topic{Verschiedenes}

Es wird über die über präzise Amtszeit des Vorstandes diskutiert. Die Versammlung kommt zu dem Konsens, dass der Vorstand 
bis zur nächsten Mitgliederversammlung im Amt ist (steht nicht explizit in der Satzung).

\topic{Verabschiedung}
Der Versammlungsleiter dankt den anwesenden Mitgliedern für ihr Kommen und beendet die Versammlung.

\end{Protokoll}
\end{document}
