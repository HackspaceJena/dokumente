\documentclass[]{scrartcl}
\usepackage[ngerman]{babel}
\usepackage[T1]{fontenc}
\usepackage[utf8]{inputenc}
\usepackage{lmodern}
\usepackage[scaled=.93]{helvet}
\usepackage{eurosym}
\usepackage{multicol}
\usepackage{scrlayer-scrpage}
\usepackage{csquotes}
\usepackage{bera}
\usepackage{berasans}
\usepackage{beraserif}
\usepackage{beramono}


\setkomafont{section}{\fontsize{13bp}{14.5bp}\selectfont\bfseries}
\setkomafont{subsection}{\fontsize{12bp}{14.5bp}\selectfont\bfseries}
\RedeclareSectionCommand[
    beforeskip=-.75\baselineskip,
    afterskip=.3\baselineskip]{section}
\RedeclareSectionCommand[
    beforeskip=-.5\baselineskip,
    afterskip=.3\baselineskip]{subsection}
% \renewcommand{\sectionheadstartvskip}{\vspace*{-1\topskip}}
% \renewcommand{\sectionheadendvskip}{\vspace*{0.2\topskip}}
\setlength{\parindent}{0pt}
\setlength{\parskip}{1ex}
\setlength{\columnsep}{1cm}
\flushbottom

%\pagestyle{myheadings}
%\markright{Protokoll der Jahreshauptversammlung 2017 des Hackspace Jena e.\,V.}
\pagestyle{scrheadings}
\ohead{Protokoll der Jahreshauptversammlung 2017 des Hackspace Jena e.\,V.}

\newenvironment{packed_enum}{
\begin{enumerate}
\setlength{\itemsep}{-5pt}
\setlength{\parsep}{0pt}
}{\end{enumerate}}
\newenvironment{packed_item}{
\begin{itemize}
\setlength{\itemsep}{-5pt}
\setlength{\parsep}{0pt}
}{\end{itemize}}

\newenvironment{eingerueckt}{
\begin{addmargin*}[3ex]{-3ex}
\begin{minipage}{\linewidth}
}{
\end{minipage}
\end{addmargin*}}


\makeatletter
\newenvironment{AbsatzHaengend}[1]{\par\noindent\@hangfrom{\textbf{#1}}\ignorespaces}
                                  {\par}
\makeatother

\newcommand{\qbi}{Jens Kubieziel}
\newcommand{\berndf}{Bernd Kampe}
\newcommand{\berndm}{Bernd Hüsing}
\newcommand{\adrian}{Adrian Pauli}
\newcommand{\gecko}{Gecko}
\newcommand{\joe}{Jonas Melzer}
\newcommand{\hanna}{Johanna Schell}

\begin{document}

\thispagestyle{plain}
\textbf{\large{Protokoll der Jahreshauptversammlung des Hackspace Jena
e.\,V.}}
\vspace{5mm}
\begin{multicols}{2}

\begin{AbsatzHaengend}{Ort: }Vereinsräume des Hackspace Jena,\\Krautgasse 26, 07743 Jena\end{AbsatzHaengend}
\begin{AbsatzHaengend}{Zeit: }09.\,Dezember~2017 \\15:23 Uhr -- 19:15 Uhr\end{AbsatzHaengend}
\begin{AbsatzHaengend}{Anwesend: }13~Mitglieder und 2~Gäste\end{AbsatzHaengend}
\begin{AbsatzHaengend}{Versammlungsleitung: }\qbi{}\end{AbsatzHaengend}
\begin{AbsatzHaengend}{Protokoll: }\berndm{}\end{AbsatzHaengend}

\section{Begrüßung}
15:23 Uhr eröffnet der amtierende Vorsitzende \qbi{} die Versammlung mit der
Begrüßung der Teilnehmer. Anwesend sind 13~Mitglieder und 2~Gäste.

\section{Wahl der Versammlungsleitung}
Die Anwesenden betrauen \qbi{} mit der Leitung der Versammlung und \berndm{}
mit der Protokollierung.

\section{Vorstellung der Tagesordnung}
\qbi{} verliest die Punkte der Tagesordnung und die zu behandelnden Themen, wie
sie auch im Wiki zu finden sind.
\begin{packed_enum}
\item Begrüßung
\item Wahl der Versammlungsleitung
\item Vorstellung der Tagesordnung
\item Feststellung der ordentlichen Einladung
\item Abnahme des Protokolls von 2016
\item Rechenschaftsberichte
\item Entlastung des Vorstandes
\item Wahl des neuen Vorstandes
\item Wahl der Kassenprüfer
\item Sonstige Themen
\item Verabschiedung
\end{packed_enum}
Die Tagesordnung und Themen werden mit einer Stimme Enthaltung angenommen.

\section{Feststellung der ordentlichen Einladung}
Es wird festgestellt, daß alle Mitglieder ordnungsgemäß per E-Mail bzw.
Brief über den Termin informiert wurden. Die Versammlung ist beschlußfähig.

\section{Annahme des Protokolls von 2016}
Das Protokoll der Jahreshauptversammlung 2016 liegt in vielfacher Ausführung
allen zur Einsicht vor und wird in offener Abstimmung mit einer Stimme
Enthaltung angenommen.

\section{Rechenschaftsberichte}
\subsection{Schatzmeister}
\berndf{} gibt seinen Rechenschaftsbericht. Der Verein finanziert
sich in erster Linie durch die Mitgliedsbeiträge und den Verkauf der
Getränke. Dieser ist wichtig, da die Mitgliedsbeiträge nicht ausreichen, um
die Fixkosten zu decken. Das Repariercafe macht plus. Bei der Kassenprüfung
wurde festgestellt, daß die Kasse ein vierstelligen Minusbetrag aufweist. Da
die Einnahmen aus dem Getränkeverkauf anscheinend nicht dem entspechen, was
bei der eingekauften Menge zu erwarten wäre, besteht der Verdacht des
Diebstahls aus der Getränkekasse.

Es wird angemerkt, daß die Spendenkasse der Langen Nacht der Wissenschaft
noch nicht verbucht wurde. \qbi{} wirft ein, daß die Entlastung des Vorstandes
am Minus scheitern könnte.

Die Festkosten werden jetzt per SEPA-Lastschriftverfahren automatisch
eingezogen. Der im letzten Jahr vorgeschlagene Wechsel der Versicherung
wurde geprüft und ist inzwischen vollzogen. Die Steuererklärung zum Erhalt
der Gemeinnützigkeit für den Zeitraum 2013 bis 2016 wurde erledigt.
Ansonsten der übliche Bürokram.
    
Späterer Nachtrag: Das Repariercafe hat heute den Antrag für den Bus
gestellt.

\subsection{Bericht der Kassenprüfer}
\joe{} gibt einen Bericht zur Kassenprüfung. Die Führung der Kasse  erfolgte
ordnungsgemäß.

\subsection{Schriftführer(in)}
\hanna{} gibt ihren Rechenschaftsbericht. Der Haupteil der Arbeit war
bürokratischer Kram, Berichte schreiben und gelegentliche Vermittlung bei
Konflikten.

\subsection{Vorsitzender}
\qbi{} gibt einen sehr ausführlichen Rechenschaftsbericht. Neben dem üblichen
Bürokram hat er sich im letzten Jahr um folgende Sachen gekümmert.
\begin{packed_item}
\item Abarbeiten von Tickets im OTRS
\item Loomio für Abstimmungen wurde eingerichtet
\item Sicherheitsbeauftragten gefunden
\item Besuch des Vereinsforum der Bürgerstiftung
\item Besichtigungen von Räumlichkeiten für das Repariercafe
\item Ausstehende Mitgliedsbeiträge -- 4 säumige Mitglieder wurden
ausgeschlossen, 1 Mitglied hat bezahlt, 6 Mitglieder befinden sich noch in
der Schwebe. Generell werden die Mitgliedsbeiträge aber pünktlich bezahlt.
\end{packed_item}
\berndf{} merkt an, daß die Mitgliederzahl seit zwei Jahren ziemlich
konstant ohne Karteileichen bei 40 liegt.
\begin{packed_item}
\item Gespräch mit einem Richter bezüglich der Vergabe von Bußgeldern an
gemeinnützige Vereine. Dieser Prozeß stockt momentan.
\item Der Verein hat ein Fördermitglied. Es wird an einem anderen
potentiellen Sponsor gearbeitet.
\item Zusammenarbeit mit der Kinderinitiative und anderen zwecks
Fördergeldern.
\item Versuch Fördermittel aus Lottogeldern zu erhalten
\item Erarbeitung der \enquote{7 Thesen für ein bürgerfreundliches digitales Jena}
\item Umzug des Server des Vereins
\item Cryptoworkshop mit dem Arbeitskreis kritischer Juristen
\item Workshop mit Kindern der Jena-Planschule
\end{packed_item}

\section{Entlastung des Vorstandes}
Fragen zu den Rechenschaftsberichten gab es keine. Es wird einstimmig
beschlossen, die Abstimmung über die Entlastung des Vorstandes in einer
offenen Abstimmung vorzunehmen. Zusätzlich wird mit einer Stimmenenthaltung
dafür plädiert, den Vorstand gemeinsam zu entlasten.

\begin{eingerueckt}
\begin{tabular}{lr}
Stimmen für eine Entlastung: & 9 \\
Stimmenenthaltungen: & 4 \\
Stimmen gegen eine Entlastung: & 0
\end{tabular}
\end{eingerueckt}

Damit ist der bisherige Vorstand entlastet.

\textsl{Die Versammlung pausiert 5 Minuten.}

\section{Wahl des neuen Vorstandes}
\adrian{} übernimmt die Funktion des Wahlleiters. Das Finden eines Kandidaten
für das Amt des Schatzmeisters gestaltet sich etwas schwierig. Schlußendlich
stehen folgende Kandidaten zur Wahl:
\begin{packed_item}
\item Vorsitzender: \gecko{}, \qbi{}
\item Schatzmeister: \adrian{}
\item Schriftführer: \qbi{}
\end{packed_item}
Die Kandidaten stellen sich und ihre Ziele vor. Im Anschluß erfolgt in
geheimer Abstimmung die Wahl.

\subsection{Vorsitzender}
\begin{eingerueckt}
\begin{tabular}{lr}
Stimmen für \qbi{}: & 1 \\
Stimmen für \gecko{}: & 9 \\
Stimmenenthaltungen: & 1 \\
Ungültige Stimmen: & 2
\end{tabular}
\end{eingerueckt}

\gecko{} nimmt die Wahl an.

\subsection{Schatzmeister}
\begin{eingerueckt}
\begin{tabular}{lr}
Stimmen für \adrian{}: & 11 \\
Stimmenenthaltungen: & 2 \\
Gegenstimmen: & 0
\end{tabular}
\end{eingerueckt}

\adrian{} nimmt die Wahl an.

\subsection{Schriftführer}
\begin{eingerueckt}
\begin{tabular}{lr}
Stimmen für \qbi{}: & 9 \\
Stimmenenthaltungen: & 3 \\
Gegenstimmen: & 1
\end{tabular}
\end{eingerueckt}

\qbi{} nimmt die Wahl an.

\section{Wahl der Kassenprüfer}
Als Kandidaten stellen sich \joe{} und Tim Schell zur Verfügung. In einer
offenen Abstimmung werden beide einstimmig in diese Funktion berufen. Beide
nehmen die Wahl an.

\vspace{1em}
\textsl{Die Versammlung pausiert 5 Minuten. Ein Gast verläßt die
Versammlung.}

\section{Sonstige Themen}

\subsection{Anpassung der Mitgliedsbeiträge}
\berndf{} erläutert die Gründe hierfür.
\begin{packed_item}
\item Hohe Festkosten (ca. \EUR{680} pro Monat)
\item Stromkosten wegen über 1000~KW mehr gestiegen
\end{packed_item}

Der Vorschlag wären \EUR{20} für reguläre Mitglieder, sowie \EUR{10} für
eine ermäßigte Mitgliedschaft. Es gibt Bedenken Leute mit geringen
finanziellen Mitteln dadurch auszuschließen. Anderseits ist der Beitrag
verhältnismäßig gering und die Satzung würde in Härtefällen ausdrücklich
Abweichungen davon ermöglichen. Eine offene Abstimmung erbringt das
Ergebnis:

\begin{eingerueckt}
\begin{tabular}{lr}
Stimmen für eine Erhöhung: & 13 \\
Stimmenenthaltungen: & 0 \\
Stimmen gegen eine Erhöhung: & 0
\end{tabular}
\end{eingerueckt}

Damit ist die Beitragserhöhung beschlossen und tritt ab Januar~2018 in
Kraft.
    
\subsection{Teilnahme an den Chemnitzer Linuxtagen mit einem Stand}
\qbi{} erläutert den Hintergrund. Es gibt Bedenken, daß der Aufwand den Nutzen
übersteigt. Generell gibt es aber eine zustimmende Haltung zu dem Vorschlag.
\joe{}, Lakritzguru, Felix und \adrian{} bilden den Kern eines Orga-Teams dafür.

\subsection{Aktivitäten zur Erlangen von Fördergeldern}
Aufgrund der Finanzlage gibt es überlegungen zur Erlangung von
Fördermitteln. Dazu stehen folgende Überlegungen im Raum:
\begin{packed_item}
\item Gründung eines ERFA-Kreises. Zum einem wegen der Thematischen Nähe,
zum anderen würde es dem Verein Fördermittel des CCC erschließen. 
\item Fördermittel aus dem Topf der Soziokulturförderung der Stadt Jena.
Hier wären zweckgebundene Fördermittel für einzelne Veranstaltungen möglich.
\item Die Züblin~AG und die Sparkasse vergeben Fördermittel.
\item Weiterhin gibt es wohl die Möglichkeit der Bewerbung auf Preisgelder.
\end{packed_item}
Generell müssen solche Anträge gut begründet werden. Um den finanziellen
Handlungsspielraum so groß wie möglich zu halten, sollte sich der Verein
darum bemühen, daß die Gelder möglichst wenig zweckgebunden sind.

\subsection{Erneute Abwägung der Zielsetzung des Hackspaces als ERFA-Kreis des
CCC}
\qbi{} erläutert die Bedingungen, welche an einen ERFA-Kreis gebunden sind. Wir
würden einige dieser Bedingungen erfüllen, aber bei weitem nicht alle.
Besonders nicht, daß der gesamte Vorstand Mitglied im Chaos Computer Club seien
müßte. Faktisch würde ein ERFA-Kreis die Gründung des CCC Jena bedeuten.
Eine so enge Bindung an den CCC ist nur bedingt erwünscht.

\subsection{Bekanntmachung der Türschließanlage}
Aufbau und Funktion der Türschließanlage ist inzwischen bekannt. Dank und
Applaus für die Macher der Anlage.
    
\subsection{Quo Vadis Verein?}
\enquote{Was wollen wir machen?}
\begin{packed_item}
\item regelmäßiges Junghacking
\item Veranstaltungen im Bereich Amateufunk
\item Klassischen LPD \enquote{nur} im zweiten Halbjahr. Überlegungen im ersten
Halbjahr eine Art weiterführenden Kurs zu veranstalten.
\item \enquote{Coding Corner} nur unregelmäßig, aber in Verbindung mit
einem Vortrag zum Thema
\end{packed_item}
Generell wird festgestellt, daß die Planung noch Möglichkeiten zur
Verbesserung bietet. In Zukunft sollten Roadmaps und Checklisten für
Veranstaltungen erstellt werde. Wir sollten längerfristig planen.

Für den Betrieb des Hackspaces ist ein gewisser Grundstock an Geräten
notwendig. Wenn es andere Makerspaces in Jena gibt, soll eine Kooperation
mit diesen angestrebt werden.

\subsection{Strafrechtlich relevante oder hackerunethische Dinge im
Hackspace}
Es wurden anscheind gestohlene Dinge in den Hackspace gebracht. Als
Mitwisser macht man sich strafbar. Wie wollen wir damit umgehen?
\begin{packed_item}
\item Es müssen Konsequenzen folgen
\item Diese sollen im Ernstfall bis zum Ausschluß der Person reichen
\item Erste Maßnahme soll ein Gespräch mit der Person sein
\end{packed_item}
Im September wurde das erste mal festgestellt, daß Geld aus der Kasse fehlt.
Es wird erwogen Strafanzeige gegen Unbekannt zu stellen. Die andere Frage
ist, welche vorbeugenden Maßnahmen wir für die Zukunft ergreifen wollen.
Dazu gibt es den Vorschlag Beträge ab einer bestimmten Höhe nicht mehr bar,
sondern per Überweisung zu begleichen. Das wird als unpraktisch abgelehnt.
Genereller Konsens ist, daß wir keine unsozialen oder illegalen Aktionen im
Hackspace wollen.

\subsection{Brand im Keller}
Seit dem Brand im Keller kommt ein recht penetranter Gestank aus dem
Lüftungsschacht in der Toilette. Der Vermieter soll um eine Reinigung des
Lüftungsschachtes gebeten werden.

\subsection{Delegieren von Aufgaben des Vorstandes}
Es kam die Frage auf, ob der Vorstand Teile seine Aufgaben an Mitglieder
delegieren darf. Es wird festgestellt, daß dem nichts im Wege steht.

\section{Verabschiedung}
Aus Zeitgründen wird die Versammlung 19:15 Uhr für beendet erklärt. Die
noch ausstehenden Punkte sollen zu einem nicht näher festgelegten Zeitpunkt
abgearbeitet werden. Lakritzguru lädt für den kommenden Mittwoch zum 5-jährigem
Jubiläum der Brettspielrunde ein.

\end{multicols}

\vspace{10mm}
\textbf{\textsf{\large{Der neue Vorstand}}}

Der Vorstand des Hackspace Jena e.\,V. setzt sich seit dem 09.\,Dezember~2017 
wie folgt zusammen:

\vspace{3mm}
\begin{eingerueckt}
\begin{tabular}{ll}
Vorstandsvorsitzender: & \gecko{} \\
Schatzmeister: & \adrian{} \\
Schriftführer: & \qbi{}
\end{tabular}
\end{eingerueckt}

\vspace{10mm}
\textbf{\textsf{\large{Für die Richtigkeit des Protokolls}}} \\[2mm]
\hfill
\begin{minipage}[t]{20mm}
\vspace{22mm}
\hrule
\vspace{2mm}
\small{Datum}
\end{minipage}
\begin{minipage}[t]{50mm}
\vspace{22mm}
\hrule
\vspace{2mm}
\small{\qbi{}\\\textsl{Versammlungsleiter}}
\end{minipage}
\hfill
\begin{minipage}[t]{20mm}
\vspace{22mm}
\hrule
\vspace{2mm}
\small{Datum}
\end{minipage}
\begin{minipage}[t]{50mm}
\vspace{22mm}
\hrule
\vspace{2mm}
\small{\berndm{}\\\textsl{Protokollführender}}
\end{minipage}
\hfill
\\[8mm]
Jena, den 10. Dezember 2017

\end{document}
