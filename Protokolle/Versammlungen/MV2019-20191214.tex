\documentclass[DIV=calc,parksip=half*]{scrartcl}
\usepackage[ngerman]{babel}
\usepackage[T1]{fontenc}
\usepackage[utf8]{inputenc}
\usepackage{eurosym}
\usepackage{multicol}
\usepackage{scrlayer-scrpage}
\usepackage{csquotes}
\usepackage{bera}
\usepackage{berasans}
\usepackage{beraserif}
\usepackage{beramono}
\usepackage[colorlinks,breaklinks]{hyperref}
\usepackage{booktabs}
\usepackage{tabularx}
\usepackage{paralist}

\pagestyle{scrheadings}
\ohead{Protokoll der Mitgliederversammlung des Hackspace Jena
  e.\,V. vom 14.\,Dezember~2019}

\makeatletter
\newenvironment{AbsatzHaengend}[1]{\par\noindent\@hangfrom{\textbf{#1}}\ignorespaces}
                                  {\par}
\makeatother

\newcommand{\qbi}{Jens Kubieziel}
\newcommand{\fiveop}{Philipp Schäfer}
\newcommand{\jonny}{Jonny Maik Müller}
\newcommand{\thomas}{Thomas Lotze}
\newcommand{\ckv}{Sebastian Wagner}
\newcommand{\quanten}{quanten}
\newcommand{\isi}{Isi}


\begin{document}

\thispagestyle{plain}
\section*{Protokoll der ordentlichen Mitgliederversammlung des Hackspace Jena
e.\,V.}

\begin{AbsatzHaengend}{Ort: }Vereinsräume des Hackspace Jena,\\Krautgasse 26, 07743 Jena\end{AbsatzHaengend}
\begin{AbsatzHaengend}{Zeit: }14.\,Dezember~2019 \\15:04~Uhr -- 16:56 Uhr\end{AbsatzHaengend}
\begin{AbsatzHaengend}{Anwesend: }16~stimmberechtigte Mitglieder und ein~Gast\end{AbsatzHaengend}
\begin{AbsatzHaengend}{Versammlungsleitung: }\qbi{}\end{AbsatzHaengend}
\begin{AbsatzHaengend}{Protokoll: }\quanten\end{AbsatzHaengend}

\section{Begrüßung}
Um 14:19~Uhr eröffnet der amtierende Vorstandsvorsitzende \qbi{} die Versammlung mit der
Begrüßung der Teilnehmer. Anwesend sind 16~Mitglieder und ein Gast.
\section{Wahl des Versammlungsleiter und Protokollführer}
Die Anwesenden betrauen \qbi{} in einer offenen Abstimmung mit der Leitung der Versammlung. Die
Abstimmung erfolgt einstimmig. Weiterhin wird \quanten{}
mit der Protokollierung betraut. Hierbei stimmen 15~Mitglieder zu. Es gibt eine Enthaltung.

\section{Abstimmung der Tagesordnung}
Die geplanten Punkte der Tagesordnung werden verlesen. Es gibt eine kurze Diskussion, ob ein Tagesordnungspunkt \enquote{Sonstiges} eingefügt werden soll. Dieser Vorschlag wird wieder zurückgezogen. Die Tagesordnung wird mit einer Enthaltung und keiner Gegenstimme angenommen.
\begin{compactenum}
\item Begrüßung
\item Wahl der Versammlungsleitung und des Protokollführers
\item Abstimmung der Tagesordnung
\item  Genehmigung der Protokolle von der Versammlung vom 7.\,September~2019
\item  Rechenschaftsbericht des Vorstands
\item  Bericht der Kassenprüfer
\item Abstimmung über Entlastung des Vorstands
\item Wahl des neuen Vorstands
\item Wahl der Kassenprüfer
\item Abstimmung zur Satzungsänderung zu den Kompetenzen der Beisitzer
\item Abstimmung zur Wahl von Beisitzer
\item Abstimmung zur Satzungsänderung bezüglich Stimmübertragungen zu Mitgliederversammlungen
\item Verabschiedung
\end{compactenum}


\section{Genehmigung der Protokolle von der Versammlung vom 7.\,September~2019}


Es wird festgestellt, dass die Mitglieder satzungsgemäß eingeladen worden
sind. Sämtliche Mitglieder wurden rechtzeitig per E-Mail oder gegebenenfalls per
Brief informiert.  Die Versammlung ist beschlussfähig.

Die Protokolle der Versammlung aus dem September 2019 sind zum Zeitpunkt der
Versammlung bei Github veröffentlicht. 

Es erfolgt eine Erläuterung der Unterschiede der öffentlich verfügbaren\footnote{\url{https://kraut.space/verein:dokumente} sowie
\url{https://github.com/HackspaceJena/dokumente/tree/master/Protokolle/Versammlungen}} und offiziellen Version des Protokolls. Letzteres geht an den Notar sowie an das Vereinsregister und enthält Namen, Adresse und Geburtstdatum der Vorstände sowie eine Liste der anwesenden Mitglieder. Daher wird die öffentlich verfügbare Version um diese personenbezogenen Daten gekürzt.

Es gibt keine Fragen oder Anmerkungn zum alten Protokoll, woraufhin eine offene Abstimmung folgt. Mit 11~Stimmen dafür, keiner Gegenstimme und 5~Enthaltungen wird das Protokoll genehmigt.
\section{Rechenschaftsbericht des Vorstands}

Alle drei anwesenden Vorstandsmitglieder legen Rechenschaft ab. 

Der Vorstand übernahm im September die Amtsgeschäfte. Dabei wurde versucht, Zugriff auf das Konto zu bekommen und den neuen Vorstand im Vereinsregister einzutragen. Insbesondere durch die Einführung der PSD2-Richtlinie und der damit verbundenen Umstellung der Authentifikation gab es einige Schwierigkeiten, sich den Kontozugriff zu sichern.
Weiterhin wurden die schriftlichen Unterlagen übergeben und vom neuen Vorstand gesichtet.

Den Vorstand erreichte eine Auskunftsanfrage nach der DSGVO. Diese wurde geprüft und beantwortet. Daneben fand eine Prüfung zum Umsetzungsstand der DSGVO statt. Einige noch zu erledigende Punkte wurden identifiziert und umgesetzt bzw. es wird an der Umsetzung gearbeitet.

Der Verein hat bei der Commerzbank Jena ein Konto für die Mietkaution. Die Bank wollte die Unterlagen vervollständigen. Der alte Vorstand sandte bereits die Satzung hin. Der Betreuer bei der Commerzbank bestand auf einer vollständigen Mitgliederliste des Vereins. Jens führte daraufhin mehrere Gespräche mit dem Betreuer und wir konnten uns einigen, dass die Anzahl der Vereinsmitglieder für seine Zwecke reicht. Die Anzahl wurde an die Commerzbank gemeldet. Eine Mitgliederliste oder ähnliche personenbezogene Daten wurden nicht weitergegeben und unseres Erachtens besteht hierfür auch keine Notwendigkeit.

Eine wesentliche Aufgabe ist die Kommunikation über das OTRS (E-Mails an die Adresse <office@…>). Es wurden in der Amtszeit knapp 400 Tickets eröffnet. Etwas mehr als die Hälfte entfällt auf Spam. Der Rest sind »normale E-Mails«. Dort wurden verschiedene Anliegen an den Verein herangetragen. Einige wurden an die Mailingliste weitergeleitet, andere mit einzelnen Mitgliedern oder nur im Vorstand besprochen und dann beantwortet.

Auf Antrag der außerordentlichen Mitgliederversammlung hat der Vorstand einen Satzungsänderungsvorschlag bzgl. der Beisitzer eingebracht. 

Es wurden seit Amtsantritt des Vorstands vier neue Vereinsmitglieder begrüßt und im selben Zeitraum mindestens fünf frühere Austritte in den Unterlagen nachgetragen. Wir haben säumige Beiträge eingefordert und erste Nachzahlungen erhalten.
Wir haben die Getränkebar wieder aktiver verwaltet und im Zweckbetrieb dadurch höhere Einnahmen erzielt. Es wurden ebenfalls neue Getränke ausprobiert und die Snackbar wiederbelebt.
Der Entwurf von Vereins-T-Shirts und die Organisation von Vorträgen wurden angestoßen.

Auf Nachfrage ergänzte der Vorstand zu sachbezogenen Spenden, dass es eine kleine Spende für ein Projekt während des Camp19 gab und weitere Spenden, die bevorzugt für bestimmte Dinge verwendet werden sollten.
Das Reparier-Cafe hat auch mehrere sachbezogene Spenden erhalten.

Auf eine Nachfrage zur Finazsituation wurde berichtet, dass der Verein aktuell rund \EUR{2700} verfügbar hat.
\section{Bericht der Kassenprüfer}
Am 9.\,Dezember~2019 erfolgte die Kassenprüfung durch \fiveop{} und  \ckv. Auf der
Mitgliederversammlung gibt \fiveop{} einen Bericht über die Kassenprüfung ab. Für die Barkasse wurden Buchungen (Filter: 05.09.2019--31.12.2020, Konto 0 - Barkasse, Projekt Verein/Raum) und Stand geprüft. Zu allen Buchungen waren Belege vorhanden. Die Barkasse enthielt 33 Cent zu viel (im Vergleich zu 10 Cent
zu viel bei der letzten Prüfung). Eine Überweisung von 500€ auf das Konto des Hackspaces, nach Entnahme von 500€ in Bar durch den Schatzmeister, war zum Prüfungszeitpunkt noch nicht auf dem Konto angekommen. Den Überweisungsbeleg des Schatzmeisters haben wir gesehen.

\thomas{} sagt \enquote{und die Buchung ist inzwischen auch auf dem Konto angekommen.} und wir haben es gesehen.

Für die Barkasse des Repariercafés wurden Buchungen (Filter:
05.09.2019--31.12.2020, Konto 3 - Repariercafe) und Stand geprüft.
Zu allen Buchungen waren Belege, soweit möglich, vorhanden. Die Barkasse enthielt genau den erwarteten Betrag.


Für das Girokonto wurden alle Buchungen außer die Mitgliedsbeiträge (Filter:
07.09.2019--31.12.2020, Konto 3183912 - Ethikbank) geprüft. Es fehlen Belege zu
folgenden Vorgängen:
\begin{itemize}
 \item 2 Buchungen zu Bußgeldern wegen Falschparkens des Busses
 
 \thomas{} sagt: \enquote{Ist bei Oda angefragt, wobei sie meinte, daß wir Papier haben müßten. Muß ich noch mal suchen.}
 
 \item Notarkosten in Höhe von \EUR{118,52} für die Eintragung der letzten zwei Vorstände

\thomas{} sagt: \enquote{Such ich auch noch mal.}
\end{itemize}
Rechnungen zum Strom konnten nicht eingesehen werden, da ein Login zur Webseite des Lieferanten fehlt.

Die erste geprüfte Buchung ist vom 4.9.2019 mit der Nummer \#3737,
die Letzte die \#3947 vom 9.12.2019.
Der Bestand auf dem Kautionskonto wurde nur in den Büchern geprüft.

Die Entlastung des Vorstands wird empfohlen.

\section{Abstimmung über Entlastung des Vorstands}

Es gab keine Fragen zu den einzelnen Rechenschaftsberichten.
Die Kassenprüfung erfolgte mit der Empfehlung, den Vorstand zu entlasten.
Über die Entlastung der Vorstandsmitglieder wird einzeln geheim abgestimmt.

%\begin{eingerueckt}
  \begin{tabularx}{\linewidth}{Xr}
    \toprule
    Vorstandsvorsitz: & \textsc{\qbi{}}\\
    Stimmen für eine Entlastung: & 13 \\
    Stimmenthaltungen: & 3 \\
    Stimmen gegen eine Entlastung: & 0\\
    \midrule
    Schriftführer: & \textsc{\jonny{}}\\
    Stimmen für eine Entlastung: & 14 \\
    Stimmenthaltungen: & 2 \\
    Stimmen gegen eine Entlastung: & 0\\
    \midrule
    Schatzmeister: & \textsc{\thomas{}}\\
    Stimmen für eine Entlastung: & 14 \\
    Stimmenthaltungen: & 2 \\
    Stimmen gegen eine Entlastung: & 0\\
    \bottomrule
\end{tabularx}
%\end{eingerueckt}
Damit ist der bisherige Vorstand entlastet.

Um 15:22 verlässt Bastian Stein die Versammlung. Nun sind noch 15 stimmberechtigte Mitglieder anwesend.

\section{Wahl des neuen Vorstands}

\quanten{} nimmt die Funktion des Wahlleiters ein.

\qbi{} wird als Vorstandsvorsitzenden vorgeschlagen. Weiterhin kandidieren
\jonny{} als Schriftführer und \thomas{} als Schatzmeister. Weitere Kandidaten
gibt es nicht.

Die zur Wahl stehenden Personen sind:
\begin{compactenum}
\item Vorsitzender: \qbi{}
\item Schatzmeister: \thomas{}
\item Schriftführer: \jonny{}
\end{compactenum}

Die Wahl erfolgt in geheimer Abstimmung.

\subsection*{Ungültige Stimmen}

Eine der abgegebenen Stimmen ist ungültig.

\begin{tabularx}{\linewidth}{Xr}
Abgegebene Stimmen: & 15 \\
  Gültige Stimmen: & 14 \\
  Ungültige: & 1\\
\end{tabularx}

\subsection{Vorsitzender}
%\begin{eingerueckt}
\begin{tabularx}{\linewidth}{Xr}
Stimmen für \qbi{}: & 12 \\
  Stimmenthaltungen: & 0 \\
  Gegenstimmen: & 2\\
\end{tabularx}
%\end{eingerueckt}
\qbi{} nimmt die Wahl an.

\subsection{Schatzmeister}
%\begin{eingerueckt}
\begin{tabularx}{\linewidth}{Xr}
Stimmen für \thomas{}: & 12\\
Stimmenthaltungen: & 1 \\
Gegenstimmen: & 1
\end{tabularx}
%\end{eingerueckt}
\thomas{} nimmt die Wahl an.


\subsection{Schriftführer}
%\begin{eingerueckt}
\begin{tabularx}{\linewidth}{Xr}
Stimmen für \jonny{}: & 11\\
Stimmenthaltungen: & 1\\
Gegenstimmen: & 2
\end{tabularx}
%\end{eingerueckt}
\jonny{} nimmt die Wahl an.

Der Wahlleiter kontrolliert die im Protokoll vermerkten Stimmen.

\section{Wahl der Kassenprüfer}

In einer offenen Abstimmung wird mit 14 Ja-Stimmen, keiner Nein-Stimme und einer Enthaltung beschlossen zwei Kassenprüfer*innen zu wählen.

Als Kandidaten stellen sich \fiveop{} und \isi{} zur Verfügung. Über beide
Kandidaten wird im Block geheim abgestimmt.

\subsection{Kassenprüfer}
%\begin{eingerueckt}
\begin{tabularx}{\linewidth}{Xr}
Stimmen für \fiveop{} und \isi{}: & 13\\
Stimmenthaltungen: & 1\\
Gegenstimmen: & 1
\end{tabularx}
%\end{eingerueckt}

\fiveop{} und \isi{} nimmt die Wahl an.
\section{Abstimmung zur Satzungsänderung zu den Kompetenzen der Beisitzer}

Es wurde in der Einladung zur MV ein Antrag zur Satzungänderung veröffentlicht, der sich auf §9 Vorstand Absatz 1 bezieht. (Die vorgeschlagene Änderung ist fett markiert.)

\begin{quote}
§9 Vorstand

 1. Der Vorstand besteht aus mindestens drei ordentlichen Mitgliedern: dem Vorstandsvorsitzenden, dem Schatzmeister und dem Schriftführer. Des Weiteren können bis zu drei Beisitzer in den \textbf{erweiterten} Vorstand gewählt werden. \textbf{Die Beisitzer beraten den Vorstand und sind berechtigt an allen Vorstandssitzungen teilzunehmen, besitzen aber kein Stimmrecht im Vorstand.} Es kann auf Wunsch der Mitgliederversammlung auf eine Wahl der Beisitzer verzichtet werden.
\end{quote}

Mit folgender Begründung: 
\begin{quote}
Wie auf der außerordentlichen Mitgliederversammlung beschlossen, soll die Rolle der Beisitzer im Vorstand eindeutig festgelegt werden. Nach aktueller Satzung erfolgt die Wahl in den Vorstand (§9.1) und für Beschlüsse werden  eine zwei-drittel-Mehrheit benötigt, wobei jedes Vorstandsmitglied stimmberechtigt ist (§9.10). Zählt man Beisitzer als Teil des Vorstandes, die diesbezüglich stimmberechtigt wären, könnte dies die Entscheidungsfindung im Vorstand deutlich komplizierter machen. Die Änderung soll diesen Sachverhalt eindeutig klären.
\end{quote}

Der alte Vorstand erklärt, nach Rücksprache mit dem Notar, dass nach der aktuellen Satzung Beisitzer vollständige Mitglieder des Vorstands sind und auch volles Stimmrecht haben. Beisitzer unterscheiden sich vorallem dadurch von Gästen einer Vorstandssitzung, dass sie von der MV gewählt werden. Des Weiteren müssen genügend Beisitzer anwesend sein, da diese mit in die Berechnung der Beschlussfähig des Vorstands zählen. Auch müssten Beisitzer mit Stimmrecht bei einer MV entlastet werden.

Es wird über die Funktion von stimmberechtigten Beisitzern als demokratisches Element diskutiert. Der Vorstand muss den Verein nach außen vertreten, wobei unklar ist, welche Rolle hier die stimmberechtigten Beisitzer spielen. Auch würde die Mehrheitsfindung mit bis zu 6 Stimmberechtigten im Vorstand, der ja möglichst ein Team darstellen solle, schwerer. Der Aufwand einer Satzungsänderung wird genannt. 
Ein GO Antrag auf sofortige Schließung der Rednerliste wird durch Stimmengleichheit (5/5/5) abgelehnt.
Es wird weiterhin über die Gewichtung/Stellung der Beisitzer zum Vorstand diskutiert und auf die Möglichkeit der MV hingewiesen, auf die Wahl von Beisitzer zu verzichten. Von einem Mitglied wird sich ein offenes Meinungsbild gewünscht, welches eine generelle, aber nicht einstimmige Zustimmung zu der Satzungsänderung signalisiert.

Es folgt eine geheime Abstimmung zur Satzungsänderung um 16:34. Der Antrag wird mit 14 Ja-Stimmen, einer Gegenstimme und keiner Enthaltung angenommen.
\section{Abstimmung zur Wahl von Beisitzern}

Es wird offen darüber abgestimmt, ob auf Beisitzer verzichtet werden soll. Mit 6 Ja-Stimmen, 7 Gegenstimmen und zwei Enthaltungen spricht sich die MV dafür aus, Beisitzer zu wählen.

Der alte Vorstand weißt darauf hin, dass Beisitzer persönliche Informationen einsehen, bei jeder Vorstandssitzung dabei seien dürfen und letzteres die Planung dieser verlangsamen könnte. Ein Mitglied merkt an, dass die MV in der nun kommenden Wahl immer noch keinem Beisitzer eine genügendende Mehrheit geben könnte und es somit zu keinem Beisitzer kommen könnte. 

Es gibt folgende Vorschläge:
\begin{itemize}
 \item \isi{}
 \item \quanten{}
 \item Lukas
 \item \fiveop{}
\end{itemize}

Es folgt eine geheime Abstimmung. Jedes Mitglied kann für jeden Kandidaten Vorschlag mit Ja/Nein/Enthaltung stimmen. Es sind die drei Beisitzer, die die meisten Ja-Stimmen gewählt, zusätzlich muss jeder Kandidat mehr als die Häfte Ja-Stimmen erhalten haben.

\begin{tabularx}{\linewidth}{Xr}
    \toprule
    \textsc{\isi{}}\\
    Ja-Stimmen & 11 \\
    Enthaltungen & 2 \\
    Nein-Stimmen & 2 \\
    \midrule
    \textsc{\quanten{}}\\
    Ja-Stimmen & 11 \\
    Enthaltungen & 1 \\
    Nein-Stimmen & 3 \\
    \midrule
    \textsc{Lukas}\\
    Ja-Stimmen & 5 \\
    Enthaltungen & 2 \\
    Nein-Stimmen & 8 \\
    \midrule
    \textsc{\fiveop{}}\\
    Ja-Stimmen & 12 \\
    Enthaltungen & 1 \\
    Nein-Stimmen & 2 \\
    \bottomrule
\end{tabularx}
\par

\isi{}, \quanten{} und \fiveop{} nehmen die Wahl an.


Um 17:03 verlässt \fiveop{} die Versammlung. Nun sind noch 14 stimmberechtigte Mitglieder anwesend.

\section{Abstimmung zur Satzungsänderung bezüglich Stimmübertragungen zu Mitgliederversammlungen}
Es wurde in der Einladung zur MV ein Antrag zur Satzungänderung veröffentlicht, der einen vierten Absatz zur §8 Mitgliederversammlung hinzufügt.

\begin{quote}
§8 Mitgliederversammlung

4. Jedes anwesende ordentliche Mitglied kann, zusätzlich zu seiner eigenen Stimme, die Stimme maximal eines weiteren ordentlichen Mitglieds in Vertretung übernehmen.
Die Vollmacht bedarf der Schriftform und muss dem Versammlungsleiter übergeben werden.
Eine Einschränkung der Vollmacht durch den Bevollmächtigenden ist nicht möglich.
\end{quote}

Mit folgender Begründung: 
\begin{quote}
Zur Zeit ist nicht geregelt, wie mit übertragenen Stimmen umgegangen wird. Laut Satzung des HackspaceJena E.V. hat jedes Mitglied genau nur eine Stimme [§3]. Somit ist das MV Protokoll anfechtbar.
\end{quote}

Der Vorstand erläutert, dass mit der aktuellen Satzung keine Stimmübertragung möglich ist. Da es bei der außerordentlichen Mitgliederversammlung im September keine knappen Abstimmungen gegeben hat, ist damit kein Problem entstanden. Ein Mitglied berichtet, dass er im Vorfeld der MV Fragen von meheren Mitgliedern nach Stimmübergabe abgelehnt hat. Über die Anzahl an auf eine Person übertragbaren Stimmen gibt es verschiedene Meinungen, da zum einen der Wille von meheren Personen geäußert werden könne und zum anderen sich damit das direkte Organ der MV undemokratisch werden könnte. Dabei wird auf den vorliegenden Antrag verwiesen, der nicht geändert werden kann. Das Problem einer möglichen Übernahme eines Vereins durch eine solche Regelung wird besprochen, durch die Regelung in der aktuellen Form aber als eher Abstrakt bezeichnet.

In einer geheimen Abstimmung wird mit 13 Ja-Stimmen, keiner Gegenstimme und einer Enthaltung die Änderung angenommen.


\section{Verabschiedung}
Der neue Vorstand bedankt sich für die Beteiligung und erklärt die Versammlung um 17:28~Uhr für  beendet.

%\end{multicols}

%\vspace{10mm}
\clearpage{}
\section*{Der neue Vorstand}
Der Vorstand des Hackspace Jena e.\,V. setzt sich seit der MV am 14.\,De\-zem\-ber~2019
wie folgt zusammen:

% \vspace{3mm}

%\begin{eingerueckt}
\begin{tabularx}{\textwidth}{lX}
  Vorstandsvorsitzender: & \qbi{} \\
  Schatzmeister: & \thomas{} \\
  Schriftführer: & \jonny{}\\
\end{tabularx}
%\end{eingerueckt}

\section*{Liste der Anwesenden zur ordentlichen Mitgliederversammlung 2019}


Das Originalprotokoll, welches an Notar und Vereinsregister geht, enthält die
Namen aller anwesenden Mitglieder und ggf. die Anzahl der Gäste.


\vspace{10mm}
\textbf{\textsf{\large{Für die Richtigkeit des Protokolls}}} \\[2mm]
\hfill
\begin{minipage}[t]{20mm}
\vspace{22mm}
\hrule
\vspace{2mm}
\small{Datum}
\end{minipage}
\begin{minipage}[t]{40mm}
\vspace{22mm}
\hrule
\vspace{2mm}
\small{\qbi{}\\\textsl{Versammlungsleiter}}
\end{minipage}
\hfill
\begin{minipage}[t]{20mm}
\vspace{22mm}
\hrule
\vspace{2mm}
\small{Datum}
\end{minipage}
\begin{minipage}[t]{40mm}
\vspace{22mm}
\hrule
\vspace{2mm}
\small{\quanten{}\\\textsl{Protokollführer}}
\end{minipage}
\hfill
\\[8mm]
Jena, den 8.\,September~2019

\end{document}
