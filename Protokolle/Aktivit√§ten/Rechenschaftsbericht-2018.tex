\documentclass[ngerman]{scrartcl}
\usepackage[utf8]{inputenc}
\usepackage[T1]{fontenc}
\usepackage{lmodern}

\usepackage{xcolor}
\usepackage{hyperref}
\definecolor{darkblue}{rgb}{0,0,.5}

\hypersetup{pdftex=true, colorlinks=true, %
	breaklinks=true, linkcolor=black, %
	urlcolor=darkblue}

\usepackage[ngerman]{babel}
%\usepackage{enumitem}
\usepackage{marvosym}

% Euro
\usepackage{eurosym}

% Einheiten und Zahlen korrekt setzen
\usepackage[binary-units=true]{siunitx}
\sisetup{locale = DE, detect-all}
\DeclareSIUnit{\EUR}{\text{\euro{}}}

\usepackage{booktabs}
\usepackage{tabularx}

\usepackage{fixltx2e}
\usepackage[final,babel]{microtype} % Verbesserung der Typographie
\usepackage{paralist}

\usepackage{bera}
\usepackage{berasans}
\usepackage{beraserif}
\usepackage{beramono}
\usepackage{csquotes}
%\setitemize{itemsep=0pt}

\title{Rechenschaftsbericht}
\subtitle{Hackspace Jena e.\,V.}
\author{%
	Gecko Gecko (Vorsitzender)\\
	Jens Kubieziel (Schriftführer)\\
	Adrian Pauli (Schatzmeister)
}
\date{10.\,Dezember~2017 bis 01.\,Dezember~2018}

\begin{document}

\maketitle{}
\newpage

\tableofcontents{}

\newpage{}

\section{Mitgliederenwicklung}

Zum Stichtag, dem 24.\,November~2018, hat unser Verein 56~Mitglieder sowie ein Fördermitglied.
Seit der letzten Jahreshauptversammlung haben wir 13~Mitglieder begrüßt und 6~Mitglieder verabschiedet.

\section{Finanzen}

Im Zeitraum vom 03.\,Dezember~2017 bis zum 24.\,November~2018 erhielt unser Verein Einnahmen von \num{26420,46}~\euro{} und tätigte Ausgaben von \num{18869,37}~\euro{}.
Daraus ergibt sich ein Zuwachs von \num{7551,09}~\euro{}.


\subsection{Ideeller Bereich}
\label{sec:ideeller_bereich}

Im ideellen Bereich gab es in diesem Zeitraum folgende Einnahmen:
\begin{compactitem}
\item Mitgliedsbeiträge in Höhe von \num{9334}~\euro{}
\item \num{1341,92}~\euro{} Spenden
\item \num{10000}~\euro{} Fördermittel für den Reparier-Café Bus
\end{compactitem}
Insgesamt sind das Einnahmen von \num{20675,92}~\euro{}.

Die Ausgaben in diesem Zeitraum für Miete, Internet sowie die Abschläge für Nebenkosten betragen \num{8001,18}~\euro{}.
\num{20}~\euro{} haben wir an den FSFE e.\,V. gespendet.
\num{5775,21}~\euro{} wurden für den Reparier-Café Bus ausgegeben.
Für sonstige Sachen wurden \num{1785,9}~\euro{} ausgegeben.
Dies sind zum Teil Ausstattungsgegenstände und Verbrauchsmaterialien wie Visitenkarten, Reinigungsmittel, Müllbeutel usw.
Außerdem wurden Werkzeuge und Bauteile für die Werkstatt und die CNC Fräse angeschafft und ein Digitaloszilloskop.
Gesamt sind das Ausgaben in diesem Bereich von \num{15583,29}~\euro{}.

Der ideele Bereich hat somit einen Überschuss von \num{5093,62}~\euro{} eingebracht.
\newpage
\subsection{Zweckbetrieb}
\label{sec:Zweckbetrieb}
Aus den Verkäufen von Getränken und Snacks ergaben sich Einnahmen von \num{5141,22}~\euro{}, wobei für \num{2987,95}~\euro{} Waren eingekauft haben.
Damit ergibt sich ein Überschuss von \num{2153,27}~\euro{}, der für Finanzierungen im ideellen Bereich verwendet werden kann.

Das Reparier-Café nahm durch ihre Veranstaltungen \num{603,32}~\euro{} ein und gab für Ersatzteile \num{62,65}~\euro{} aus.
Damit ergibt sich ein Überschuss von \num{540,67}~\euro{}, der für Finanzierungen im ideellen Bereich verwendet werden kann.

Insgesamt hat damit unser Zweckbetrieb einen Überschuss von \num{2693,94}~\euro{} erwirtschaftet.

\subsection{Zweckgebundene Spenden}
\label{sec:zweckgebundene_spenden}


\begin{table}[h]
        \centering
        \begin{tabular}{l|r|r|r}
        \toprule
        \textsc{Projekt} & \textsc{Eingang} & \textsc{Ausgang} & \textsc{Stand} \\
        \midrule
        Freifunk & \num{0}~\euro{} & \num{20,99}~\euro{} & \num{5,01}~\euro{} \\
        Reparier-Café & \num{0}~\euro{} & \num{232,51}~\euro{} & \num{650,96}~\euro{} \\
        Werkstattbus Reparier-Café & \num{10000}~\euro{} & \num{5885,21}~\euro{} & \num{4114,79}~\euro{} \\
        Tor-Relay & \num{0}~\euro{} & \num{19,84}~\euro{} & \num{220,16}~\euro{} \\
        Veranstaltung Förderung Ehrenamt 2018 & \num{300}~\euro{} & \num{0}~\euro{} & \num{300}~\euro{} \\
\bottomrule
        \end{tabular}
        \caption{Eingänge/Ausgänge Zweckgebundene Spenden}
        \label{table:spenden}
\end{table}

\subsection{Kontoführung}
\label{sec:Kontoführung}
Für die Kontoführung wurden \num{236,48}~\euro{} aufgewendet.

\subsection{Aktueller Kontostand}

\begin{table}[h!]
        \centering{}
        \begin{tabular}{l|r}
        \toprule
        \textsc{Konto} & \textsc{Kontostand} \\
        & \textsc{am 24.\,11.\,2018} \\
        \midrule
        Barkasse & \num{105}~\euro{} \\
        Reparier-Café Barkasse & \num{176,85}~\euro{} \\
        Kautionskonto & \num{1668,11}~\euro{} \\
        Girokonto & \num{9035,06}~\euro{}\\
        \bottomrule
        \end{tabular}
\caption{Übersicht der Konten}
\end{table}

%\section{Veranstaltungen}
\newpage
\section{Vereinsaktivitäten}

Ein großer Teil der Vereinstätigkeiten ergibt sich aus der
Bereitstellung der Infrastruktur. So haben sich regelmäßige offene Runden
etabliert, in denen themenbezogen gearbeitet wird. Für die
einzelnen Veranstaltungen haben sich Freiwillige aus dem Verein
gefunden, die sich um die Organisation kümmern.

\begin{table}[h]
  \centering{}
	\begin{tabularx}{\textwidth}{l|X}
          \toprule
		\textsc{Name} & \textsc{Turnus} \\ \midrule
		Elektronikrunde   & wöchentlich \\
		Chaostreff        & wöchentlich \\
		Spieleabend       & zweiwöchentlich \\
		Linux User Group  & wöchentlich \\
		Gaming am Freitag & zweiwöchentlich \\
        Plenum            & monatlich \\
		Thuringiafurs Stammtisch & monatlich \\
		Saturday Make Session    & zweiwöchentlich \\
		Reparier-Café     & monatlich \\
\bottomrule
\end{tabularx}

\caption{Aktivitäten}
\end{table}

\subsection{Elektronikrunde}

Die Elektronikrunde trifft sich wöchentlich im Krautspace.
Sie unterstützt bei Elektronikprojekten, Fehlersuche,
Reparaturen und fördert den Erfahrungsaustausch.
Der Verein stellt dabei einen großen Teil der Werkzeuge und
Verbrauchsmaterialien bereit.
Dieses Jahr haben wir unsere Werkstatt um eine CNC-Fräse erweitert und
ein neues Digital-Oszilloskop angeschafft.

Es wurden auch die Veranstaltungen vorbereitet: Löt-Workshop, Maustüröffnertag.

\subsection{Chaostreff und monatliches Plenum}

Der Chaostreff ist unsere wöchentlich stattfindende themenoffene Runde.
An diesem Tag steht der Raum Mitgliedern und Gästen zur freien Verfügung.
Neben den gut besuchten Treffen gab es Vorträge und studentische 
Einführungsveranstaltung.

Die Themen der Vorträge lagen im Bereich Informationstechnologie, 
Computersicherheit, Netzpolitik, Datenschutzes, Verschlüsselung, CAD,
Hackathon.

Einmal im Monat gab es ein Plenum. Hier wurden vereinsinterne Themen und Projekte besprochen.

\subsection{Spieleabend -- Gesellschaftsspielerei}

Am Spieleabend werden alle zwei Wochen anspruchsvolle Brett- und
Kartenspiele mit unterschiedlichen Spielkonzepten gespielt.
Es wurden aktuelle Spiele von Spielemessen präsentiert.

\subsection{GNU/Linux User Group}

Die Linux-User-Group trifft sich wöchenlich um Themen der freien Software
zu behandeln. Die Veranstaltung wurde gut besucht, es gab Vorträge. Es
konnte vielen Gästen bei unterschiedlichen Computerproblemen geholfen werden.

\subsection{Gaming am Freitag}

Beim Gamingstammtisch geht es um Computerspiele — egal auf welcher
Plattform, ob gekauft oder selbst geschrieben. Die Schwerpunkte sind
Game Design und die Auswirkungen des Spielens auf Spieler und
Gesellschaft.

\subsection{Reparier-Café}

Seit 31.\,Juli~2014 ist das Reparier-Café ein fester Bestandteil unseres Vereins.
Es leistet Hilfe bei der Reparatur von Gebrauchsgegenständen.
Seine monatlichen Veranstaltungen werden sehr gut besucht. In diesem Jahr wurden
von Fördergeldern ein Linienbus angeschafft, der zu einem mobilen Werkstattbus
umgebaut wird.

\subsection{Funkwoche}

Dieses Jahr luden wir zweimal zur Funkwoche um uns mit Themen rund um CB-Funk und Satellitenkommunikation
zu beschäftigen. Es gab Vorträge von Mitgliedern und Gästen, sowie Prototypenbau und Diskussionsrunden.
In diesem Rahmen haben wir auch an den Spacetalks der ESA teilgenommen.

\subsection{Saturday Make Session}

Zweimal im Monat stehen die Räume des Hackspace jedem Bastler offen. Unter
Anleitung kann gebohrt, gefräst, gefeilt, gelötet und programmiert werden.

\subsection{Öffentlichkeitsarbeit}
Als Verein haben wir zusätzlich auch an mehreren Veranstaltungen mitgewirkt:
\begin{compactitem}
	\item Stand bei den Chemnitzer Linux Tagen 2018
	\item Vorträge zur ALOTA (Einführungsveranstaltung zum Studienanfang)
	\item Workshop zum Bau von Alarmanlagen mit Kindern zum Maustüröffnertag
	\item Lötworkshop im Rahmen des Informatik-Sommercamp
\end{compactitem}


\section{Tätigkeitsberichte des Vorstandes}

\subsection{Gecko}

Gecko hat sich als Vorstandsvorsitzender mit folgenden Themen beschäftigt:

\begin{compactitem}
    \item Vernetzung mit anderen Vereinen, Institutionen und Initiativen:
    \begin{compactitem}
        \item Cellu l'art Festival Jena e.\,V. 
        \item FSU FSR Informatik
        \item Kulturschlachthof Jena
        \item Offenes Jena (Open Knowledge Foundation)
        \item Lichtwerkstatt
        \item FSU Stura
        \item Leuchtturm Jena
        \item CCC e.\,V.
        \item Maschinenraum Weimar
        \item Studierendenwerk Jena
        \item Deutsches Luft- und Raumfahrtzentrum (DLR)
    \end{compactitem}
    \item Teilnahme an vernetzenden und öffentlichkeitswirksamen Veranstaltungen
    \begin{compactitem}
        \item Chaos Communication Congress (34C3)
        \item Chemnitzer Linuxtage 2018
        \item Easterhegg 2018
        \item Geekend Dezentrale Leipzig
        \item Haxogreen Luxembourg
        \item Gulaschprogrammiernacht
        \item MRMCD 2018
        \item Datenspuren 2018
        \item OpenRheinRuhr
    \end{compactitem}
    \item Teilhabe an der Organisation von Events
    \begin{compactitem}
        \item Cellu l'art Kurzfilmnacht im Krautspace
        \item Etablierung regelmäßiger Plenen
        \item Pitch-Vortrag Coding Da Vinci Ost
        \item WDR Maustüröffnertag
        \item Saturday Make Session
        \item Vorbereitung Drohnenstammtisch
    \end{compactitem}
    \item Pflege der Werkstatt
    \item Koordination von Vereinsfragen
    \item Hausmeister und Ansprechpartner für Raumpflege
    \item Getränke für die Bar
    \item Anschaffung Bus 
    \begin{compactitem}
        \item Koordination mit Repariercafe
        \item Unterzeichnung des Kaufvertrages
        \item Abnahme des Busses
    \end{compactitem}
\end{compactitem}

\subsection{Jens}

\subsubsection{Bürokratie}


Jens' Rolle im Vorstand des Vereins wechselte von der des Vorstands in die des
Schriftführers. Im Dezember 2017 hat er die Außendarstellung des Vereins
angepasst, so dass die Änderungen im Vorstand auch veröffentlicht sind. Leider
gab es seitens eines Vorstandsmitglieds keine Bereitschaft, der
Impressumspflicht Folge zu leisten. Dieser Umstand wurde von Jens im Laufe des
Jahres mehrfach (leider erfolglos) angesprochen.

Für die Koordination der Vorstandsmitglieder gibt es die Mailingliste
\href{https://lstsrv.org/mailman/listinfo/hackspace-office}{hackspace-office@lstsrv.org}. Die
Mitglieder wurden geändert, dass diese den aktuellen Stand
repräsentieren. Weiterhin wurde ein neuer Account im OTRS für gecko angelegt.

Für künftige Wechsel des Vereinsvorstands wurde eine
\href{https://kraut.space/hswiki:anleitungen:vorstandswechsel}{Checkliste}
angelegt. Diese soll helfen, alle notwendigen Schritte zu planen und
durchzuführen.

Über das gesamte laufende Jahr bearbeitete Jens Tickets im OTRS, antwortete auf
Anfragen und leitete wichtig erscheinende Tickets an Mitglieder, die
Mailingliste oder andere weiter.

\subsubsection{Ausschluss eines Mitglieds}

Bei der letzten Mitgliederversammlung im Jahr 2017 gab es Diskussionen um
Verluste im Zweckbetrieb. Dabei entstand der Verdacht, dass es ein Mitglied,
welches systematisch Geld aus der Kasse entwendet.

Der neue Vorstand besprach den Sachverhalt auf einer Sitzung und entschied sich
den Sachverhalt zu beobachten. Kurze Zeit später konnten weitere unberechtigte
Geldentnahmen festgestellt werden und das Mitglied wurde zur Rede
gestellt. Der Vorstand entschied aufgrund des Gesprächs einstimmig, das
betreffende Mitglied auszuschließen und ein Hausverbot auszusprechen. Über das
Hausverbot wurden einzelne Mitglieder informiert, um dies auch effektiv
durchsetzen zu können. In der Folge gab es einige Versuche seitens des
Ex-Mitglieds Zugang zum Space zu bekommen. Dies wurde verwehrt. Seit mehreren
Monaten wird der Ausschluss und das Hausverbot akzeptiert und es konnten keine
neuen Versuche festgestellt werden.

\subsubsection{Gemeinnützigkeit und Freifunk}

Das Finanzamt schrieb den Verein an und hinterfragte, die
Gemeinnützigkeit. Hintergrund war eine Vermutung bezüglich der verwalteten
Hardware. Dieses Missverständnis wurde zusammen mit dem Vertreter des Freifunk
Jena aufgeklärt und ein Schreiben an das Finanzamt Jena verfasst. Darauf gab es
keinerlei Rückmeldung. Daher wird dies als erledigt betrachtet.

\subsubsection{Rundfunkbeitrag}

Jens fand heraus, dass gemeinnützige Vereine zunächst nur ein Drittel des
Rundfunkbeitrages zahlen müssen. Unter bestimmten Umständen fällt dieser Beitrag
sogar auf Null. Dies wurde mit dem ARD ZDF Deutschlandradio Beitragsservice
geklärt und wir erreichten eine Reduktion des Rundfunkbeitrages.

\subsubsection{Twitter und andere soziale Medien}

Der Verein hat einige Twitter"=Konten. So gibt es
u.\,a. \href{https://twitter.com/ksraumstatus}{@KSRaumStatus}. Für dieses Konto
hatten wir zunächst keine Zugangsdaten. Diese wurden wieder
besorgt. Mittlerweile twittert ein Bot wieder den Status des Raumes und auf
Twitter ist zu sehen, ob der Raum besetzt ist oder nicht.

Daneben half Jens hin und wieder Tweets abzusetzen bzw. Sachen zu
retweeten. Allerdings steckt horn hier wesentlich mehr Zeit und Aufwand
hinein. Dafür gebührt ihm der Dank des Vorstands.

Der Verein betreibt auch eine Präsenz auf Facebook. Hin und wieder antwortete
Jens auf Anfragen. Die Hauptarbeit liegt jedoch bei Vine. Jens dankt für das
diesbezügliche Engagement.

Weiterhin gab es eine Repräsentanz auf Quitter. Die Seite verschwand leider im
Laufe des Jahres. Daher ist auch unser Account nicht mehr erreichbar.

Es wäre sehr wünschenswert, wenn der Verein einen Account bei Mastodon hätte. Im
Sinne der Dezentralisierung wäre auch der Betrieb einer eigenen Instanz schön.

\subsubsection{Fehlende Beiträge}


Jens nahm sich einer Aufgabe aus der letzten Vorstandsperiode an. Er sprach
einige Mitglieder an, die mit der Zahlung im Rückstand waren. Dabei konnte
einige Erfolge verzeichnet werden. Es gibt jedoch immernoch einige Personen, die
teils Jahre im Rückstand sind. Hier sollte der neue Vorstand weiter mit Umsicht
agieren, säumige Mitglieder mahnen und durchaus bei Nicht"=Erfolg ausschließen.

\subsubsection{Koordination von Vorträgen}

Jens half, einige Vorträge im Verein zu koordinieren und hielt auch
Vorträge. Eine der nennenswerten Veranstaltungen waren die bundesweiten
Aktionstage für Netzpolitik und Demokratie. Hier wurde zunächst eine Umfrage
über die Teilnahme initiiert. Nachdem die Mehrweit dafür gestimmt hatte, wurden
mehrere Vorträge organisiert. Einen davon hielt Jens selbst.

\subsubsection{Tor-Relay}

Vor längerer Zeit rief Jens die Mitglieder auf, für den Betrieb eines
Tor"=Servers zu spenden. Dabei war die Maßgabe, dass es kein Exitserver sein
solle. Vielmehr sollte zunächst Erfahrung mit einem Mittelknoten oder einer
Bridge gesammelt werden. Hierfür sind Gelder vorhanden (siehe
\autoref{sec:zweckgebundene_spenden}). Einer der gemieteten vServer stellte sich
als ungeeignet für andere Aufgaben heraus. Daher wurde dieser als Server für
eine Tor"=Bridge eingerichtet. Dieser läuft weitgehend wartungsfrei und benutzte
zwischen \SI{2}{\gibi\byte} und \SI{1,1}{\tebi\byte} pro Monat.

\subsubsection{Dezentrale Dienste}

Der Verein betreibt diverse Dienste für Vereinszwecke. Dabei ist es sinnvoll,
einige der Ressourcen auch nahestehenden Vereinen oder Organisationen zur
Verfügung zu stellen. Daher wurden einige Dinge im Chaosumfeld zur Benutzung
angeboten. Daraus resultierte eine Nutzung des Abstimmungswerkzeugs
\href{https://loomio.kraut.space/}{Loomio} durch die Haecksen. Dies ist ein
Zusammenschluss von weiblichen Mitgliedern des CCC. Diese nutzen unser Loomio
für deren interne Abstimmungen.

\subsubsection{Datenschutz}

Mit der Anwendbarkeit der EU"=Datenschutz"=Grundverordnung prüfte Jens einige
relevante Dinge im Verein und formulierte die
\href{https://kraut.space/datenschutz}{Datenschutzerklärung} für den Verein.


\subsection{Adrian}
Adrian hat sich als Schatzmeister und Vorstandsmitglied mit Folgendem beschäftigt:
\begin{compactitem}
	\item Finanzverwaltung und Planung
	\begin{compactitem}
		\item Buchführung
		\item Rechnungen bezahlen
		\item Unterlagen prüfen und abheften
		\item Unterstützung Kassenprüfung
		\item Zuwendungsbescheinigungen erstellt
	\end{compactitem}
	\item Mitgliederverwaltung
	\begin{compactitem}
		\item Mitglieder persönlich begrüßt und in die Verwaltung aufgenommen
		\item Fragen von Mitglieder bezüglich ihrer Beiträgen beantwortet
		\item Anschreiben säumiger Mitglieder
		\item Verabschiedung von Mitgliedern
	\end{compactitem}
	\item Bar mit Getränken und Süßigkeiten:
	\begin{compactitem}
		\item Planung der Warenbeschaffung
		\item Einkauf von Snacks
		\item Absprachen mit Verantwortlichen
	\end{compactitem}
	\item Erstellung des Rechenschaftsberichts
	\item Verwaltung von Post und Postfach
	\item Treffen und Absprachen im Vorstand
	\item Absprachen mit Projektverantwortlichen
	\item Schreiben der Quartalsberichte
	\item Mitorganisation und Unterstützung von:
	\begin{compactitem}
		\item Stand bei Chemnitzer Linux Tage 2018
		\item Maustüröffnertag
		\item Lötworkshop
		\item ALOTA
		\item Anschaffung Bus für Reparier-Café
		\item Plenen
	\end{compactitem}
	
\end{compactitem}

\end{document}

%%% Local Variables:
%%% mode: latex
%%% TeX-master: t
%%% End:
