\documentclass[ngerman]{scrartcl}
\usepackage[utf8]{inputenc}
\usepackage[T1]{fontenc}
\usepackage{lmodern}

\usepackage{xcolor}
\usepackage{hyperref}
\definecolor{darkblue}{rgb}{0,0,.5}

\hypersetup{pdftex=true, colorlinks=true, %
	breaklinks=true, linkcolor=black, %
	urlcolor=darkblue}

\usepackage[ngerman]{babel}
%\usepackage{enumitem}
\usepackage{marvosym}

% Euro
\usepackage{eurosym}

% Einheiten und Zahlen korrekt setzen
\usepackage{siunitx}
\sisetup{locale = DE, detect-all}
\DeclareSIUnit{\EUR}{\text{\euro{}}}

\usepackage{booktabs}
\usepackage{tabularx}

\usepackage{fixltx2e}
\usepackage[final,babel]{microtype} % Verbesserung der Typographie
\usepackage{paralist}

\usepackage{bera}
\usepackage{berasans}
\usepackage{beraserif}
\usepackage{beramono}
\usepackage{csquotes}
%\setitemize{itemsep=0pt}

\title{Rechenschaftsbericht}
\subtitle{Hackspace Jena e.\,V.}
\author{%
	Jens Kubieziel (Vorsitzender)\\
	Johanna Schell (Schriftführer)\\
	Bernd Kampe (Schatzmeister)
}
\date{27.11.2016 bis 09.12.2017}

\begin{document}

\maketitle{}

\tableofcontents{}

\newpage{}

\section{Mitgliederenwicklung}

Zum Stichtag, dem 02.\,Dezember~2017, hat unser Verein 48~Mitglieder sowie ein Fördermitglied.
Seit der letzten Jahreshauptversammlung haben wir 6~Mitglieder begrüßt und 6~Mitglieder verabschiedet.

\section{Finanzen}

Im Zeitraum vom 26.\,November~2016 bis zum 02.\,Dezember~2017 erhielt unser Verein Einnahmen von \num{11613,70}~\euro{} und tätigte Ausgaben von \num{14904,27}~\euro{}.
Daraus ergibt sich ein Verlust von \num{3290,57}~\euro{}.


\subsection{Ideeller Bereich}
\label{sec:ideeller_bereich}

Im ideellen Bereich gab es in diesem Zeitraum folgende Einnahmen:
\begin{compactitem}
\item Mitgliedsbeiträge in Höhe von \num{6506}~\euro{}
\item \num{255}~\euro{} Spenden
\end{compactitem}
Insgesamt sind das Einnahmen von \num{6761,00}~\euro{}.

Die Ausgaben in diesem Zeitraum für Miete, Internet sowie die Abschläge für Nebenkosten betragen \num{8704,63}~\euro{}.
\num{200}~\euro{} hat das Reparier-Café für den Aufbau einer Fahrradwerkstatt an den ADFC Jena gespendet.
Für sonstige Sachen wurden \num{2303,15}~\euro{} ausgegeben.
Dies sind zum Teil Ausstattungsgegenstände und Verbrauchsmaterialien wie Visitenkarten, Reinigungsmittel, Müllbeutel usw.
Außerdem wurden Werkzeuge und Bauteile für die Werkstatt angeschafft.
Gesamt sind das Ausgaben in diesem Bereich von \num{11207,78}~\euro{}.

\subsection{Zweckbetrieb}
\label{sec:Zweckbetrieb}
Aus den Verkäufen von Getränken und Snacks ergaben sich Einnahmen von \num{3585,08}~\euro{}, wobei für \num{3058,50}~\euro{} Waren eingekauft haben.
Damit ergibt sich ein Überschuss von \num{526,58}~\euro{}, der für Finanzierungen im ideellen Bereich verwendet werden kann.

Das Reparier-Café nahm durch ihre Veranstaltungen \num{1243,62}~\euro{} ein und gab für Infomaterial und Ersatzteile \num{452,43}~\euro{} aus.
Damit ergibt sich ein Überschuss von \num{791,19}~\euro{}, der für Finanzierungen im ideellen Bereich verwendet werden kann.

Insgesamt hat damit unser Zweckbetrieb einen Gewinn von \num{1317,77}~\euro{} erwirtschaftet.

\subsection{Zweckgebundene Spenden}
\label{sec:zweckgebundene_spenden}

Die Zweckbindung für die Spenden des Theremin-Projektes in Höhe von \num{95}~\euro{} wurde nun aufgehoben.

\begin{table}[h]
        \centering
        \begin{tabular}{l|r|r|r}
        \toprule
        \textsc{Projekt} & \textsc{Eingang} & \textsc{Ausgang} & \textsc{Stand} \\
        \midrule
        Linux Presentation Day und/oder Junghacker & \num{120}~\euro{} & \num{125,03}~\euro{} & \num{0}~\euro{} \\
        Theremin & \num{0}~\euro{} & \num{95}~\euro{} & \num{0}~\euro{} \\
        Freifunk & \num{0}~\euro{} & \num{0}~\euro{} & \num{26,00}~\euro{} \\
        Reparier-Café & \num{5}~\euro{} & \num{939,98}~\euro{} & \num{0}~\euro{} \\
        Junghacker & \num{0}~\euro{} & \num{380,80}~\euro{} & \num{0}~\euro{} \\
        Tor-Relay & \num{0}~\euro{} & \num{20}~\euro{} & \num{240,00}~\euro{} \\
\bottomrule
        \end{tabular}
        \caption{Eingänge/Ausgänge Zweckgebundene Spenden}
        \label{table:spenden}
\end{table}

\subsection{Aktueller Kontostand}

\begin{table}[h!]
        \centering{}
        \begin{tabular}{l|r}
        \toprule
        \textsc{Konto} & \textsc{Kontostand} \\
        & \textsc{am 02.\,12.\,2017} \\
        \midrule
        Barkasse & \num{275,26}~\euro{} \\
        Reparier-Café Barkasse & \num{103,18}~\euro{} \\
        Kautionskonto & \num{1668,11}~\euro{} \\
        Girokonto & \num{1211,63}~\euro{}\\
        \bottomrule
        \end{tabular}
\caption{Übersicht der Konten}
\end{table}

\section{Veranstaltungen}

\subsection{Regelmäßige (Vereins-)aktivitäten}

Ein großer Teil der Vereinstätigkeiten ergibt sich aus der
Bereitstellung der Infrastruktur. So haben sich regelmäßige offene Runden
etabliert, in denen themenbezogen gearbeitet wird. Für die
einzelnen Veranstaltungen haben sich Freiwillige aus dem Verein
gefunden, die sich um die Organisation kümmern.

\begin{table}[h]
  \centering{}
	\begin{tabularx}{\textwidth}{l|X}
          \toprule
		\textsc{Name} & \textsc{Turnus} \\ \midrule
		Elektronikrunde & jeden Montag ab 19:30 Uhr\\
		Offene Runde am Dienstag & jeden Dienstag ab 20 Uhr\\
		Sprechstunde Informationssicherheit & jeden ersten Dienstag im Monat ab 20 Uhr, seit Oktober 2014\\ 
		Spieleabend & jeden ungeraden Mittwoch ab 20 Uhr\\
		Linux User Group & jeden geraden Donnerstag ab 19 Uhr\\
		Freifunktreffen & nach Bedarf\\
		Lockpicking & jeden ersten Freitag im Monat ab 19 Uhr, beendet seit Juli 2014\\
		Gaming-Stammtisch & jeden ersten Freitag im Monat ab 19 Uhr, seit September 2014\\
                Plenum & nach Bedarf\\
		Öffentliche Vorstandssitzung & nach Bedarf\\
		Kochen & jeden dritten Freitag im Monat, beendet seit September 2014\\
		Thuringiafurs Stammtisch & jeden dritten Samstag im Monat ab 18 Uhr\\
		Chaoscafe / Chaostreff & jeden ungeraden Sonntag ab 16 Uhr, beendet seit März 2014\\
		Reparier-Café & monatlich seit Juli 2014\\
\bottomrule
\end{tabularx}
\caption{Regelmäßige Aktivitäten}
\end{table}

\subsubsection{Elektronikrunde}

Die Elektronikrunde trifft sich seit 2013 jeden Montag im Krautspace, um
sich konzentriert in Technikprojekte vertiefen zu können. Die
Teilnehmer helfen sich gegenseitig mit Werkzeugen, Materialien und
Wissen aus, um ihre Ideen zu verwirklichen. Der Verein stellt dabei
einen großen Teil der Werkzeuge und Verbrauchsmaterialien bereit.
Bauteile für die Schaltungen wurden durch die Teilnehmer selbstständig
organisiert.

\subsubsection{Offene Runde am Dienstag}

Jeden Dienstag gibt es die (themen-)offene Runde im Raum. Der Raum
steht zur freien Verfügung, um gemeinsam an Themen rund um
Informationstechnologie, der Computersicherheit und des Datenschutzes
zu diskutieren und zu arbeiten.

\subsubsection{Sprechstunde Informationssicherheit}

Mitte des Jahres 2014 kam die Idee zu einem Cryptofreitag auf. Dabei
sollten abweichend von den Cryptoparties nicht hauptsächlich Vorträge
gehalten werden, sondern es war angedacht sich auf die Fragen der
Besucher zu konzentrieren.  Da die potentiellen Betreuer Freitags
nicht verfügbar sind, wurde dann eine Sprechstunde für einen Dienstag
im Monat konzipiert.  Das Ziel der Veranstaltung ist es die Fragen der
Besucher zu den Themen Verschlüsselung, Privatsphäre und
Datensicherheit zu beantworten.

\subsubsection{Spieleabend -- Gesellschaftsspielerei}

In der Spielerunde werden regelmäßig Brett- und Kartenspiele zu einem
bestimmten vorher festgelegten Thema gespielt. Dabei liegt der
Schwerpunkt nicht auf den üblichen Partyspielen, sondern bei
anspruchsvollen Spielen mit unterschiedlichen Spielkonzepten. Dabei
kommen sehr viele unterschiedliche Spiele zum Zug. Teilweise werden
auch selbst entwickelte Spiele vorgestellt und ausprobiert oder neue
Spiele von Spielemessen präsentiert.

\subsubsection{Stammtisch der LUG Jena}

Der Stammtisch der Linux-User-Group Jena beschäftigt sich alle zwei
Wochen mit Themen rund um freie Software und insbesondere
GNU/Linux. Es geht dabei um den Erfahrungsaustausch und die Diskussion
aktueller Entwicklungen.

\subsubsection{Freifunktreffen}

Die wachsende Freifunkgemeinschaft in Jena trifft sich unregelmäßig im
Krautspace, um die aktuelle Entwicklung zu besprechen und
Interessierten die Konzepte hinter Freifunk zu erklären, sowie die
Software auf und hinter den von Freifunk betriebenen Knoten zu
verbessern.

\subsubsection{Gaming-Stammtisch}

Beim Gamingstammtisch geht es um Computerspiele — egal auf welcher
Plattform, ob gekauft oder selbst geschrieben. Die Schwerpunkte sind
Game Design und die Auswirkungen des Spielens auf Spieler und
Gesellschaft.

\subsubsection{Plenum bzw. offene Vorstandssitzung}

Das Plenum hat sich im Jahr 2015 von einer regelmäßigen Veranstaltung
zu einer Bedarfsveranstaltung geändert. Im Jahr 2016 wurden, um dem
Plenum etwas Leben einzuhauchen, die Vorstandssitzungen in der
Öffentlichkeit abgehalten. Leider hat dies nicht das gewünschte
Interesse nach sich gezogen und die öffentliche Vorstandssitzung ist
eingeschlafen.

\subsubsection{Reparier-Café}

Seit Mai 2014 hat eine kleine Gruppe außerhalb des Hackspace',
angefangen ein Reparier-Café zu organisieren. Dabei geht es darum,
nicht mehr funktionierende Gegenstände in Eigenregie zu reparieren.

Da die Idee auch unter Mitgliedern des Vereins viel Zustimmung fand,
haben sich einige Mitglieder daran beteiligt.

Das erste Café fand am 31.\,Juli~2014 in den Vereinsräumen statt und
war sehr gut besucht.  Später ist das Reparier-Café ein offizieller
Teil des Vereins geworden. Jeweils zum Monatsende sind alle
eingeladen, eigene Gegenstände zu reparieren oder anderen bei der
Reparatur zu unterstützen.

\subsubsection{Junghackertage}

Auf der letzten Vollversammlung kam die Idee auf, einen
Jungerhackertag zu veranstalten. Inhalt dieser Veranstaltung soll
sein, Kindern und Jungendlichen die Welt der Elektrotechnik und
Programmierung näher zu bringen.

Der erste Junghackertag fand am 23.04.2016 statt und findet seit dem in
unregelmäßig Intervallen an Samstagen statt.

\section{Tätigkeitsberichte des Vorstandes}

\subsection{Jens}

Jens hat auf seiner Benutzerseite im
Wiki\footnote{\url{https://kraut.space/user:qbi}} eine Aufzeichnung über seine
Tätigkeiten als Vorstand geführt. Die untenstehenden Punkte geben die
wichtigsten Tätigkeiten wieder. Einige kleinere Punkte wurden für diesen Bericht
ausgelassen.

\subsubsection{Bürokratie}


Zu Anfang der Tätigkeit als Vorstand standen vor allem verschiedene
bürokratische Aufgaben an. So erzeugte Jens einen neuen
GPG"=Schlüssel\footnote{\url{https://kraut.space/_media/verein:0x3d60afe3ab3fab6d.asc}}. Dieser
wurde auf Keyserver und in das Wiki hochgeladen. Jens überarbeitete die
Wiki-Seiten, beauftragte Tim den Schlüssel mit dem OTRS zu verknüpfen usw. Das
OTRS ist das zentrale Ticketsystem des Vereins. Das wird derzeit von Tim
gehostet und E-Mails, die an einige unserer E-Mail-Adressen gesendet werden,
landen hier. Jens beschaffte sich einen Zugang zum OTRS und arbeitete sich in
das System ein. Dabei fiel auf, dass eine Vielzahl unbearbeiteter oder sehr
alter Tickets existieren. Im November 2017 bearbeitete Jens mindestens
150~Tickets und versuchte, sich eine Übersicht über offene und wichtige Tickets
zu verschaffen. Das OTRS war im weiteren Verlauf des Jahres eines der zentralen
Arbeitsmittel. Dort laufen sehr viele der Anfragen zusammen. Das heißt, die
treffen ein. Unter den Mails ist zunächst Spam, der aussortiert werden
muss. Daneben wies Jens einige der Mails anderen Vorstandsmitgliedern zu und
beantworte viele andere. Weitere wurden auf Wiedervorlage gesetzt, um den Status
zu einem späteren Zeitpunkt zu prüfen.

Neben OTRS kommen immer wieder Anfragen über unsere Facebook"=Seite. Einige der
Anfragen wurden von Jens beantwortet. Diese Aufgaben wurden später von unseren
Verantwortlichen für soziale Medien übernommen.

Weiterhin legte Tim eine
Loomio"=Instanz\footnote{\url{https://loomio.kraut.space/}} an. Jens legte
daraufhin Gruppen an und administrierte die Mitglieder. Im Loomio kann über
verschiedene Belange abgestimmt werden. Dabei gibt es eine Gruppe für
Vereinsbelange, die auch nur Mitgliedern zugänglich ist. Eine andere Gruppe
richtet sich an einen größeren Kreis. Dort können auch Nichtmitglieder eine
Stimme abgeben. Die Instanz wurde für knapp zehn Abstimmungen
bzw. Meinungsbilder verwendet.

Im November~2016 wurde auch die Suche nach einem Sicherheitsbeauftragten
angestoßen, den der Verein nach §\,8 der
Geschäftsordnung\footnote{\url{https://kraut.space/_media/geschaeftsordnung.pdf}}
besitzen muss. Als Sicherheitsbeauftragter wurde schließlich
Uwe\footnote{\url{https://kraut.space/hswiki:verein:verantwortliche}} bestimmt.

Der Verein nutzte die
hackspace-jena-verein\footnote{\url{https://lstsrv.org/mailman/listinfo/hackspace-jena-verein}}
und
krautspace-flohmarkt\footnote{\url{https://lstsrv.org/mailman/listinfo/krautspace-flohmarkt}}. Beide
wurden seit längerem nicht mehr benutzt. Daher bat Jens frlan als Administrator
der Liste, diese zu löschen.

Die Bürgerstiftung lud zum Jenaer Vereinsforum ein. Bernd und Jens besuchten die
Veranstaltung und einige Vorträge auf dem Forum.

Die Zeitschrift c't nahm Kontakt zu uns auf und bot an, dass sie uns die MAKE
zuschicken. Jens koordinierte diese Sache und seither haben wir ein Freiabo der
Zeitschrift.


\subsubsection{Räume für das Reparier-Café}
\label{sec:raumrc}

Im Rahmen der Jahreshauptversammlung~2016 gab es unter anderem Diskussionen über
Räumlichkeiten. Das Reparier-Café möchte gern in größere Räume umziehen. Dazu
gab es im Dezember~2016 einen Besichtigungstermin im Löbdergraben~28. Im
folgenden nahm Jens an Diskussionen über Vor- und Nachteile der Räume
teil. Später gab es noch die Möglichkeit andere Räume zu mieten. Da die Miete
außerhalb der finanziellen Möglichkeiten des Vereins und des Reparier-Cafés
lagen, nahm Jens Kontakt zur Bürgerstiftung auf und ließ sich über
Fördermöglichkeiten beraten. Im wesentlichen wäre eine Förderung der Miete
demnach möglich. Jedoch muss nachgewiesen werden, dass die Miete innerhalb eines
bestimmten Zeitraums eigenfinanzierbar ist. Gegebenenfalls sind weitere Hürden
zu überwinden. Die Vertreterinnen und Vertreter des Reparier-Cafés wurden über
die Ergebnisse des Gesprächs informiert.

Das Reparier-Café schloss im laufenden Jahr eine Vereinbarung mit dem KSJ. Jens
prüfte den Vertrag, da der Verein Vertragspartner mit KSJ ist. 

\subsubsection{Nicht zahlende Mitglieder}

Bei der Auswertung des OTRS stellte sich heraus, dass es sehr viele Mitglieder
gibt, die mit den Beitragszahlungen im Rückstand sind. Einige haben seit
Eintritt in den Verein keine Beiträge gezahlt. Stand Mitte August~2017 betraf
dies insgesamt 11~Mitglieder, davon zahlten zwei seit 2013, drei seit 2014, zwei
seit 2015 und der Rest seit 2016 keine Beiträge mehr. Durchschnittlich betrugen
die Ausstände \num{473}~\euro{}.

Jens regte innerhalb des
Vorstands eine Diskussion zum Verbleib dieser Mitglieder an. Dabei bestand
Einigkeit, dass diese ausgeschlossen werden können. Das Vorhaben wurde ebenfalls
auf der Mailingliste des Vereins diskutiert. Jens entwarf ein Mahnschreiben und
verschickte das an alle, die seit 2013 und 2014 nicht mehr bezahlt
haben. Schließlich wurde Bernd gebeten, diejenigen, die nicht reagiert haben,
aus dem Verein auszuschließen.

\subsubsection{Verbesserung der Finanzsituation}

Neben der in \autoref{sec:raumrc} angesprochenen Beratung zu Fördermöglichkeiten
versuchte Jens, andere Finanzquellen zu erschließen oder bestehende
wiederzubeleben.

Richter am hiesigen Gericht können Bußgelder festlegen und bestimmen, an welche
Vereine die gezahlt werden. Zu diesem Zweck wollte Jens gern Kontakt mit den
betreffenden Personen herstellen und den Verein vorstellen. Dies gestaltete sich
jedoch schwieriger als gedacht. Es wurde auf verschiedenen Wegen versucht, den
Geschäftsverteilungsplan zu erhalten. Das Amtsgericht blockte die Versuche. Es
wurde lediglich eine Einsichtnahme angeboten. Die Zeiten waren für Jens bisher
nicht realisierbar. Das Vorhaben wird daher auf einen Zeitpunkt verschoben, wo
die Gelegenheit besteht, das Amtsgericht zu besuchen und Einsicht zu nehmen.

Unser Fördermitglied unterstützte den Verein sowohl mit Hardwarespenden wie auch
mit Geld. Jens erinnerte das Mitglied an offene Beträge. Diese Gelder können für
beliebige Zwecke ausgegeben werden.

Ein Mitglied regte in seiner Verwandschaft eine Spende für den Verein
an. Allerdings möchte der Verwandte ein offizielles Schreiben des Vereins
haben. Jens entwarf ein solches Schreiben. Dies muss Stand Dezember~2017 noch
verschickt werden.

Im Rahmen der Ich-kann-was-Initiative arbeitete Jens mit an einem
Förderantrag. Dieser wurde genehmigt. Jetzt müssen die entsprechenden Projekte
geplant und durchgeführt werden. Von dem Geld kann Hardware beschafft und
Veranstaltungsleiter bezahlt werden. Erste Planungstreffen dazu fanden statt.

Daneben führte Jens einige informelle Gespräche mit anderen Partnern und
Organisationen und versuchte, Informationen über Fördermglichkeiten zu
erhalten. Im Allgemeinen hängt dies vom betreffenden Projekt bzw. zu
beschaffender Hardware ab. Jens' Eindruck ist, dass es vielfältige Möglichkeiten
gibt, Anschaffungen bis \num{5000}~\euro{} zu finanzieren. Sobald es konkrete
Wünsche oder Bedarf gibt, müsste einer der Wege probiert werden.

\subsubsection{Smart City Jena}

Auf der Weihnachtsfeier im Jahr~2016 besuchte uns unter anderem Bastian
Stein. Er ist Stadtratsmitglied für Bündnis 90/Die Grünen Jena und plante im
Rahmen seines Mandats eine große Anfrage zum Thema \enquote{Digitalisierung}
einzubringen. Wir besprachen das Thema in einer größeren Runde im Krautspace
sowie bei weiteren Treffen. Die angesprochenen Punkte flossen dann in die große
Anfrage\footnote{\url{https://sessionnet.jena.de/sessionnet/buergerinfo/getfile.php?id=74543&type=do&}}
mit ein.

Weiterhin nahm Jena zu Jahresanfang an dem Wettbewerb \enquote{Digitale.Stadt by
bitkom}
teil.\footnote{\url{https://blog.jena.de/jenadigital/2017/02/16/stadtrat-beschliesst-teilnahme-am-wettbewerb-digitale-stadt/}}
Hier führte Jens Gespräche mit Michael Selle, Leiter Kommunikation der Stadt
Jena. Im Rahmen der Vorplanungen gab es eine Bürgerwerkstatt, an der ebenfalls
Personen aus dem Umfeld des Vereins
teilnahmen.\footnote{\url{https://blog.jena.de/jenadigital/2017/02/07/ergebnisse-buergerwerkstatt-3-februar-2017/}}

In weiteren Treffen mit Bastian Stein und Achim Friedland wurden sieben Thesen
für ein nachhaltiges und bürgerfreundliches digitales Jena
erarbeitet.\footnote{\url{https://kraut.space/blog:content:sieben_thesen_fu\%CC\%88r_ein_nachhaltiges_und_bu\%CC\%88rgerfreundliches_digitales_jena}}
Diese Thesen diskutierten wir mit Vertretern der Presse, was in einem Artikel in
der OTZ
mündete.\footnote{\url{http://jena.otz.de/web/lokal/suche/detail/-/specific/Jena-Sieben-Thesen-damit-digital-nicht-zur-Qual-wird-697439259}}

Nachdem die Anworten der großen Anfrage verfügbar waren, wurde diese ebenfalls
wieder ausgewertet, im Stadtrat besprochen und bei der OTZ
begleitet.\footnote{\url{http://jena.otz.de/web/lokal/suche/detail/-/specific/Jenas-IT-Risiko-ehrlich-bewertet-626061691}}

Insgesamt ist der Hackspace durch die Berichterstattung bei den Mitgliedern des
Stadtrates bekannter geworden. Auch in den Artikeln der OTZ wurde der Verein
erwähnt und könnte dadurch mehr Bekanntschaft bei den Leserinnen und Lesern der
Zeitung bekommen. Weiterhin gibt es auch das Angebot durch Reporter der OTZ, den
Verein etwas mehr vorzustellen. Hierfür müssen wir Kontakt mit den Personen
aufnehmen und einen Termin vereinbaren.

\subsubsection{Neuer Server}

Unser bisheriger Serveradmin informierte uns, dass er den Server stilllegt und
damit auch unsere virtuelle Maschine verschwindet. Daher suchte Jens einen neuen
Provider und zog den Server um. Der Umzug bzw. die Einstellungen der neuen
Maschine wurde im Wiki dokumentiert. Später fanden sich weitere Personen, die
den Server oder die installierte Software mit administrieren.

\subsubsection{Initiative kinderfreundliche Stadt e.\,V.}

Der Verein hat seit längerem eine Kooperation mit der Initiative
kinderfreundliche Stadt e.\,V. (Kinderini) sowie der Kaleidoskopschule. Diese
wurde von Micha angestoßen und zum großen Teil mit Leben gefüllt. Jens
versuchte, zunächst die Kooperationsvereinbarung zu sehen. Später verlinkte er
den Verein auf der Seite der Partner\footnote{\url{https://kraut.space/partner}}
und führte Gespräche mit der Vertreterin des Vereins, Claudia Martins.

Im Rahmen der Kooperation veranstalteten wir den Workshop \enquote{Let's
  hack}. Jens bestellte hierfür Pentabugs und koordinierte einen Teil der
Veranstaltung. Einige Personen aus dem Umfeld des Vereins halfen und sorgten für
einen erfolgreichen Ablauf des Workshops.

Später wurde der Workshop sowie andere Veranstaltungen bei den Geldgebern in
Berlin präsentiert. Wir informierten über die Planung und Umsetzung und konnten
dort mit anderen Veranstaltern ins Gespräch kommen.

\subsubsection{Weitere Veranstaltungen}

Der Verein wurde vom Bundesarbeitskreis kritischer Juristinnen und Juristen (BAKJ)
angesprochen. Diese veranstalteten deren Bundeskongress in Jena und wollten eine
Cryptoparty durchführen. Jens koordinierte Terminplanung, Umsetzung und
Helfer. Zusammen hielten wir dies als kleinen Workshop. Der BAKJ wollte uns
hierfür eine Spende überlassen.

Bereits frühzeitig versuchte Jens den Verein für die Lange Nacht der
Wissenschaften anzumelden. Dies gestaltete sich jedoch schwerer als
gedacht. Sehr oft im Prozess waren Nachfragen nötig oder Webseiten
funktionierten nicht. Schließlich konnten wir erfolgreich teilnehmen. Am Abend
waren immer wieder Besucher im Raum, die Interesse an unserem Angebot
hatten.

Vor der Teilnahme an den Alternativen Orientierungstagen (ALOTA) initiierte Jens
eine Abstimmung darüber. Im Vorfeld gab es kritische Stimmen, die die Teilnahme
ncht unbedingt befürworteten. Nachdem es überwiegend Zustimmung gab, wurde die
Veranstaltung von weiteren Mitgliedern organisiert und durchgeführt.

Die Jenaplanschule fragte an, ob sie den Hackspace in den Ferien besuchen
können. Sie stießen darauf, weil wir unter anderem auch mit dem
witelo~e.\,V. zusammenarbeiten und dort als Ansprechpartner aufgelistet
sind. Jens organisierte den Termin. Im August kamen dann zwei Gruppen in den
Raum. Joe, Bernd und Jens stellten den Verein und den 3D-Drucker vor. Wir
schraubten mit den Kindern eine Maus sowie einen Tower auf und beantworteten
viele Fragen. Die Betreuerinnen waren sehr begeistert von unserem Angebot und
kommen vermutlich in späteren Ferien auf uns zurück.

Jens initiierte einige offene Vorstandssitzungen in der Amtsperiode. Dort wurde
aktuelle Fragen besprochen. In der Zukunft wäre es zu wünschen, wenn diese
wieder in regelmäßigen Abständen durchgeführt werden.

Für den 34C3 im Dezember besorgte Jens Voucher und koordinierte die Verteilung
unter Interessierten. Auf diese Weise war es möglich, dass knapp 30~Personen
eine Karte für den Congress erhielten.


\subsubsection{Chaostreff}
\label{sec:ct}

Jens stellte irgendwann fest, dass der Eintrag des Krautspace' auf der Seite
aller Chaostreffs\footnote{\url{http://ccc.de/de/club/chaostreffs}} veraltet
ist. Gleichzeitig ist der Chaostreff, der sonntags stattfand, eingeschlafen. Auf
der anderen Seite gibt es mit der offenen Dienstagsrunde ein regelmäßiges
Treffen, welches sehr nah an der Idee eines Chaostreffs ist. Jens regte daher
ein Meinungsbild im Loomio an. Im Allgemeinen gab es Zustimmung, das
Dienstagstreffen auch in Richtung Chaostreff umzubenennen. Damit können wir den
Eintrag auf der obigen Seite aktualisieren und haben zunächst eine
Chaosrepräsentanz in Jena.

Weiterhin finden im CCC regelmäßig Regiotelkos und -treffen statt. Dort kommen
Erfas, Chaostreffs und andere Organisationen aus dem Chaosumfeld zusammen, um
verschiedene Sachen zu diskutieren. Jens begann an diesen Telkos teilzunehmen,
um den Hackspace dort auch etwas Sichtbarkeit zu verschaffen.

Weiterhin äußerte gecko das Ziel, den Verein in Richtung eines Erfas
weiterzuentwickeln. Jens informierte gecko über die Bedingungen, die für einen
Erfa des CCC vorliegen müssen. Wir beide würden das sehr begrüßen, wenn wir uns
in diese Richtung bewegen könnten. Jedoch ist dies einiger Aufwand und wird
sicher mindestens zwei oder mehr Jahre Arbeit benötigen.

\subsection{Johanna}

Johanna hat sich in ihrer Funktion als Vorstandsmitglied mit Folgendem 
beschäftigt:

\begin{compactitem}
    \item Mitorganisation und Teilnahme an Vorstandssitzungen
    \item Führen der Protokolle der Vorstandssitzungen
    \item Mitorganisation der offenen Vorstandssitzungen
    \item Führen der Protokolle der offenen Vorstandssitzungen
    \item Wahrnehmen von Terminen beim Notar
    \item Verteilung der E-Mails
    \item vorstandsinterne Absprachen und Diskussionen 
    \item Mitglieder an ihre Mitgliedsbeiträge erinnern
\end{compactitem}

\subsection{Bernd}
Bernd hat sich als Schatzmeister und Vorstandsmitglied mit Folgendem beschäftigt:
\begin{compactitem}
	\item Finanzverwaltung und Planung
	\begin{compactitem}
		\item Buchführung
		\item Rechnungen bezahlen
		\item Wiederkehrende Rechnungen vollständig auf SEPA umgestellt
		\begin{compactitem}
			\item Verbund offener Werkstätten e.V.
			\item Telefon/Internet
		\end{compactitem}
		\item Unterlagen abheften
		\item Kassenprüfung
		\item Zuwendungsbescheinigungen erstellt
		\item Steuererklärung
	\end{compactitem}
	\item Mitgliederverwaltung
	\begin{compactitem}
		\item Mitglieder persönlich begrüßt und in die Verwaltung aufgenommen
		\item Jens über säumige Mitglieder informiert
		\item Fragen von Mitglieder bezüglich ihrer Beiträgen beantwortet
		\item Rücksprache mit Schließsystemverantwortlichen
	\end{compactitem}
	\item Bar mit Getränken und Süßigkeiten:
	\begin{compactitem}
		\item Planung der Warenbeschaffung
		\item Absprachen mit Verantwortlichen
	\end{compactitem}
	\item Erstellung des Rechenschaftsberichts
	\item Verwaltung von Post und Postfach
	\item Treffen und Absprachen im Vorstand
	\item Absprachen mit Projektverantwortlichen
\end{compactitem}

Als Vereinsmitglied hat Bernd sich mit folgenden Dingen beschäftigt: 
\begin{compactitem}
	\item Vorträge in der Coding Corner
	\item Mitveranstaltung des Linux Presentation Days
	\item Skripte für Technik im Space
\end{compactitem}
\end{document}
