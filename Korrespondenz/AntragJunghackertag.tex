% Created 2016-05-13 Fr 16:21
\documentclass{scrartcl}
\usepackage[utf8]{inputenc}
\usepackage[T1]{fontenc}
\usepackage{fixltx2e}
\usepackage[ngerman]{babel}
\usepackage{microtype}
\usepackage[colorlinks,breaklinks]{hyperref}
\usepackage[type1]{libertine}
\usepackage{csquotes}
\usepackage{eurosym}


\title{Antrag zur Unterstützung des Junghackertages beim Hackspace Jena e.\,V.}
\author{Hackspace Jena e.\,V.}
\date{\today}

\hypersetup{
 pdfauthor={Jens Kubieziel},
 pdftitle={},
 pdfkeywords={},
 pdfsubject={},
 pdfcreator={Emacs 24.5.1 (Org mode 8.3.4)}, 
 pdflang={English}}
\begin{document}

\maketitle{}
\begin{abstract}
  Der Hackspace Jena e.\,V. führt im Jahr 2016 die Veranstaltungsreihe
  \enquote{Junghackertage} durch. Damit sollen Kinder und Jugendliche an den
  Umgang mit Technik herangeführt werden. Wir beantragen zur Unterstützung der
  Veranstaltungsreihe 500~\euro{} Förderung.
\end{abstract}



\section{Antrag zur Unterstützung des Junghackertages beim Hackspace Jena e.\,V.}
\label{sec:orgheadline5}
\subsection{Zusammenfassung}
\label{sec:orgheadline1}
Der Hackspace Jena e.\,V. ist ein gemeinnütziger Verein und möchte ein Angebot
für Kinder und Jugendliche schaffen. Die Junghackertage sind eine Reihe von
Veranstaltungen, bei denen es um den spielerischen Umgang mit Technik
geht. Um allen die Teilnahme zu ermöglichen, soll die Reihe auf Basis von
Spenden veranstaltet werden. Um dennoch die notwendigen Ausgaben leisten zu
können, beantragen wir eine Förderung für die Veranstaltungsreihe.
\subsection{Hackspace Jena e.\,V.}
\label{sec:orgheadline2}
Der Hackspace Jena e.\,V. wurde im Jahr 2012 in Jena gegründet und hat sich zum
Ziel gesetzt, ein Anlaufpunkt für Menschen mit Interesse an
Informationstechnologie zu sein. Dazu betreibt der Verein den Krautspace,
einen Raum in der Krautgasse~26. Der Raum dient als Treffpunkt, als
Veranstaltungsort für Vorträge und Workshops, als Ausstellungsort und vieles
mehr.

Im Verein sind derzeit 45~Mitglieder organisiert und er ist als
gemeinnützig anerkannt.

\subsection{Junghackertage}
\label{sec:orgheadline3}
Die Junghackertage ist eine Reihe von Einzelveranstaltungen, die zunächst im
Jahr 2016 stattfinden sollen. Sie richten sich an Kinder und Jugendliche und
sollen diese an den Umgang mit Technik spielerisch heranführen. Die Reihe ist
so gestaltet, dass sie verschiedene Altersgruppen anspricht.

Derzeit sind zehn Veranstaltungen in Planung. Die ersten Veranstaltungen
richten sich an Kinder im Grundschultalter. Dabei werden mit den Kindern
einfache Malroboter und Klackerlaken gebastelt. Wir zeigen, wie sich aus
einfachen Bestandteilen interessante Gegenstände basteln lassen.

In den folgenden Veranstaltungen geht es dann um elektronische Bauteile.
Aus den Bauteilen wird ein kleiner Käfer gelötet -- ein sogenannter Pentabug,
der auf Geräusche und Licht reagiert. Der Chip kann weiterhin individuell
programmiert werden. Im Rahmen der Veranstaltung steht das Löten der
Pentabugs im Vordergrund. Sie richtet sich ebenfalls an Kinder im
Grundschulalter.

Der Computer und seine Benutzung steht im Vordergrund weiterer
Veranstaltungen. So wollen wir mit allen Teilnehmern PCs aus Einzelteilen
zusammenstellen. Die Kinder und Jugendlichen sollen so den Aufbau und die
Funktionsweise von Computern kennenlernen. Nachdem grundlegende Kenntnisse in
der Hardware vermittelt wurden, soll es an die Software gehen. Die
Programmiersprache Scratch bietet einen spielerischen Einstieg in das
Programmieren. Aus grafischen Bestandteilen können einfache wie auch sehr
komplexe Programme erstellt werden. Wir wollen mit Jugendlichen erste
Schritte unternehmen und zeigen, worauf es beim Programmieren ankommt.

Zum Jahresende hin soll es Verstanstaltungen geben, die auf Webtechnologien
eingehen und solche die sich mit physikalischen Effekten beschäftigen.

Insgesamt entsteht dadurch eine Reihe von Veranstaltungen, die Kindern und
Jugendlichen Wissen in einem Bereich vermittelt, der heutzutage essenziell
erscheint und andererseits von Schulen nicht in dem erforderlichen Ausmaß
abgedeckt wird.

\subsection{Antrag}
\label{sec:orgheadline4}
Die Veranstaltung soll mittels Flyern und Plakaten, die in Schulen ausgelegt
bzw. ausgehangen werden, beworben werden. Weiterhin benötigt der Hackspace
Jena~e.\,V. Bauteile für die Veranstaltungen mit Hardware-Bezug. Ein Teil
der Mittel wurde bereits über Spenden eingeworben. Jedoch bewegt sich der
gesamte Finanzbedarf in einem Rahmen, der nicht durch Privatspenden im
Vorfeld abgedeckt werden kann.

Daher beantragen wir eine Förderung durch die Intershop-Stiftung in Höhe von
500~\euro. Dies ermöglicht es uns, die oben genannten nötigen Ausgaben
vorab zu decken und einen großen Teilnehmerkreis anzusprechen.
\end{document}
%%% Local Variables:
%%% mode: latex
%%% TeX-master: t
%%% End:
