\documentclass[a4paper,12pt,parskip=half]{scrartcl}
\usepackage[utf8]{inputenc}
\usepackage[german]{babel}
\usepackage{hyperref}

\title{Hygienekonzept Hackspace Jena e.V.}

\begin{document}
\maketitle

\section*{Hygienemaßnahmen}

Personen mit Krankheitszeichen (z. B. Fieber, Husten, Kurzatmigkeit, Luftnot, Verlust des
Geschmacks- oder Geruchssinn, Halsschmerzen, Schnupfen, Gliederschmerzen) dürfen den Raum nicht betreten.

Zur regelmäßigen Handhygiene sind Handwaschbecken mit Seifenspendern und
Einmalhandtüchern bereitgestellt. Ein Waschen der Hände beim Betreten und beim Verlassen wird empfohlen.

Benutzte Arbeitsflächen und Werkzeuge sind nach dem Benutzen mit Seife zu säubern bzw. zu desinfizieren.

Die Räume werden nach Möglichkeit häufig gelüftet, insbesondere beim Kommen und Gehen. Die Türen sollten während des Aufenthaltens offen gehalten werden.

Die Benutzung der Küche ist bis auf Weiteres untersagt.

\section*{Organisatorische Maßnahmen}

Der Zugang zum Raum ist nur Mitgliedern gestattet. Des Weiteren darf sich nur eine Person (bzw. Personen, wenn diese im Sinne des §1 Absatz 2 Infektionsschutzkonzepte der -ThürSARS-CoV-2-
MaßnFortentwVO zu einem Haushalt gehören) gleichzeitig im Raum aufhalten.

Wenn eine Person im Raum ist, sollte die Tür offen gehalten werden, um, neben einer besseren Durchlüftung, Anderen zu signalisieren, dass der Raum belegt ist. 

Die Mitglieder werden über das Hygienekonzept informiert. Außerdem wird das Hygienekonzept an der Tür und im Raum ausgehängt.

\section*{Zusammenfassung nach § 5 Infektionsschutzkonzepte der -ThürSARS-CoV-2-
MaßnFortentwVO}
\begin{enumerate}
 \item Verantwortlich: Vorstand Hackspace Jena e.V.
 \item 60 m$^2$
 \item nicht vorhanden
 \item Fenster und Türen sollten möglichst offen gehalten werden
 \item Beim Kommen und Gehen und regelmäßig während des Aufenthaltens wird gelüftet.
 \item Da nur einer Person gleichzeitig Zugang gestattet ist, ist die Begegnung von Personen ausgeschlossen.
 \item Der Zugang ist nur Mitgliedern gestattet.
 \item Es werden Möglichkeiten zur Handhygiene bereitgestellt und die Mitglieder werden direkt und per Aushang auf die Hygieneregeln hingewiesen.
 \item Nicht anwendbar, da keine Arbeitnehmer
\end{enumerate}

Das vorliegende Hygienekonzept wurde in der Vorstandssitzung des Hackspace Jena e.V. am 10.06.2020 bestätigt.

Per Mail an alle Mitglieder. Zusätzlich liegt das Konzept im Raum aus und wird an der Tür aufgehängt.

Bei Fragen usw. an \href{mailto:office@kraut.space}{office@kraut.space}.
\end{document}
