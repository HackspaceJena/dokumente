\documentclass[a4paper,12pt,parskip=half]{scrartcl}
\usepackage[utf8]{inputenc}
\usepackage[scaled=0.88]{beraserif}
\usepackage[scaled=0.85]{berasans}
\usepackage[scaled=0.84]{beramono}
\usepackage[T1]{fontenc}
\usepackage[german]{babel}
\usepackage[breaklinks,colorlinks]{hyperref}
\usepackage{siunitx}
\usepackage{vhistory}

\usepackage{mathpazo}
\linespread{1.05}
\usepackage[T1,small,euler-digits]{eulervm}
\usepackage{microtype}

\title{Hygienevorschriften zur Eindämmung der Ausbreitung von SARS-CoV-2}
\author{Hackspace Jena e.\,V.}
\date{10.\,Juni~2020}

\begin{document}
\maketitle

\section*{Einleitung}

Das vorliegende Hygieneschutzkonzept beschreibt Maßnahmen, die die Ausbreitung des SARS-Cov-2-Virus' in den Vereinsräumen des Hackspace Jena e.\,V. begrenzen oder verhindern sollen. Damit wird eine Öffnung der Vereinsräume und damit das Vereinsleben ermöglicht.

\section*{Kurzzusammenfassung}
\begin{itemize}
\item Eintritt nur für jeweils ein Mitglied
\item Zugang mit physischem Schlüssel
\item Räume lüften und Oberflächen mit Seife reinigen
\end{itemize}

\section*{Hygienemaßnahmen}

Personen mit Krankheitszeichen (z.\,B. Fieber, Husten, Kurzatmigkeit, Luftnot, Verlust des
Geschmacks- oder Geruchssinn, Halsschmerzen, Schnupfen, Gliederschmerzen) dürfen den Raum nicht betreten.

Zur regelmäßigen Handhygiene sind Handwaschbecken mit Seifenspendern und
Einmalhandtüchern bereitgestellt. Ein Waschen der Hände beim Betreten und beim Verlassen wird empfohlen.

Benutzte Arbeitsflächen und Werkzeuge sind nach dem Benutzen mit Seife zu säubern bzw. zu desinfizieren.

Die Räume werden nach Möglichkeit häufig gelüftet, insbesondere beim Kommen und Gehen. Die Türen sollten während des Aufenthalts offen gehalten werden.

Die Benutzung der Küche ist bis auf Weiteres untersagt.

\section*{Organisatorische Maßnahmen}

Der Zugang zum Raum ist nur Mitgliedern gestattet. Der Raum kann nur mit einem
physischen Schlüssel geöffnet werden. Die Schließanlage ist vorerst deaktiviert.

Im Raum darf sich nur eine Person (bzw. Personen, wenn diese zu einem Haushalt gehören) gleichzeitig im Raum aufhalten.

Wenn eine Person im Raum ist, sollte die Tür offen gehalten werden, um, neben einer besseren Durchlüftung, Anderen zu signalisieren, dass der Raum belegt ist. 

Die Mitglieder werden über das Hygienekonzept informiert. Außerdem wird das Hygienekonzept an der Tür und im Raum ausgehängt.

\section*{Angaben nach §\,5 Abs.\,3 ThürSARS-CoV-2-MaßnFortentwVO vom
  12.\,Mai~2020}

\begin{enumerate}
 \item Verantwortlich gemäß §\,2 Abs.\,2 ThürSARS-CoV-2-MaßnFortentwVO: Vorstand Hackspace Jena e.\,V.
 \item Raumgröße: \SI{60}{\square\metre}
 \item Begehbare Grundstücke unter freiem Himmel: nicht vorhanden
 \item Raumlufttechnische Ausstattung: Vier öffenbare Fenster, eine Außentür.
 \item Maßnahmen zur regelmäßigen Be- und Entlüftung: Beim Kommen und Gehen und regelmäßig während des Aufenthalts wird gelüftet.
 \item Maßnahmen zur weitgehenden Gewährleistung des Mindestabstands: Da nur einer Person gleichzeitig Zugang gestattet ist, ist die Begegnung von Personen ausgeschlossen.
 \item Maßnahmen zur angemessenen Beschränkung des Publikumsverkehrs: Der Zugang ist nur Mitgliedern gestattet.
 \item Maßnahmen zur Einhaltung der Infektionsschutzregeln: Es werden Möglichkeiten zur Handhygiene bereitgestellt und die Mitglieder werden direkt und per Aushang auf die Hygieneregeln hingewiesen.
 \item Maßnahmen zur Sicherstellung des spezifischen Schutzes der Arbeitnehmer: Nicht anwendbar, da keine Arbeitnehmer
\end{enumerate}

Das vorliegende Hygienekonzept wurde in der Vorstandssitzung des Hackspace Jena e.\,V. am 10.\,Juni~2020 bestätigt.

Per Mail an alle Mitglieder. Zusätzlich liegt das Konzept im Raum aus und wird an der Tür aufgehängt.

Bei Fragen usw. an \href{mailto:office@kraut.space}{office@kraut.space}.

\begin{versionhistory}
  \vhEntry{1.0}{10.\,06.\,2020}{Vorstand}{Initiale Version des Dokuments}
\end{versionhistory}

\end{document}

%%% Local Variables:
%%% mode: latex
%%% TeX-master: t
%%% End:
