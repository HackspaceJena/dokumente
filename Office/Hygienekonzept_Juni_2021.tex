\documentclass[a4paper,12pt,parskip=half]{scrartcl}
\usepackage[utf8]{inputenc}
\usepackage[scaled=0.88]{beraserif}
\usepackage[scaled=0.85]{berasans}
\usepackage[scaled=0.84]{beramono}
\usepackage[T1]{fontenc}
\usepackage[german]{babel}
\usepackage[breaklinks,colorlinks]{hyperref}
\usepackage{siunitx}
\usepackage{vhistory}

\usepackage{mathpazo}
\linespread{1.05}
\usepackage[T1,small,euler-digits]{eulervm}
\usepackage{microtype}

\title{Hygienevorschriften zur Eindämmung der Ausbreitung von SARS-CoV-2}
\author{Hackspace Jena e.\,V.}
%\date{10.\,Juni~2020}

\begin{document}
\maketitle

\section*{Einleitung}

Das vorliegende Hygieneschutzkonzept beschreibt Maßnahmen, die die Ausbreitung des SARS-Cov-2-Virus' in den Vereinsräumen des Hackspace Jena e.\,V. begrenzen oder verhindern sollen. Damit wird eine Öffnung der Vereinsräume und damit das Vereinsleben ermöglicht.

\section*{Kurzzusammenfassung}
\begin{itemize}
\item Eintritt für bis zu drei Mitglieder
\item Zugang mit physischem Schlüssel oder WLAN"=Schlüssel
\item Räume lüften und Oberflächen mit Seife reinigen
\item eventuell Einschränkungen bei höheren Inzidenzen
\end{itemize}

\section*{Hygienemaßnahmen}

Personen mit Krankheitszeichen (z.\,B. Fieber, Husten, Kurzatmigkeit, Luftnot, Verlust des
Geschmacks- oder Geruchssinn, Halsschmerzen, Schnupfen, Gliederschmerzen) dürfen den Raum nicht betreten.

Zur regelmäßigen Handhygiene sind Handwaschbecken mit Seifenspendern und
Einmalhandtüchern bereitgestellt. Ein Waschen der Hände beim Betreten und beim Verlassen wird empfohlen.

Benutzte Arbeitsflächen und Werkzeuge sind nach dem Benutzen mit Seife zu säubern bzw. zu desinfizieren.

Die Räume werden nach Möglichkeit häufig gelüftet, insbesondere beim Kommen und
Gehen. Die Türen sollten während des Aufenthalts offen gehalten werden.

Sofern sich mehrere Personen im Raum aufhalten, sollte auf einen entsprechenden
Abstand geachtet werden. Nach Möglichkeiten sollen Masken im Innenraum getragen werden.

Die Benutzung der Küche ist bis auf Weiteres untersagt.

\section*{Organisatorische Maßnahmen}

Der Zugang zum Raum ist allen Personen gestattet. Der Raum kann mit einem
physischen Schlüssel oder mit dem WLAN"=Schlüssel geöffnet werden. 

Im Raum dürfen sich bis zu drei Personen (bzw. Personen, wenn diese zu einem
Haushalt gehören) gleichzeitig im Raum aufhalten.

Zur Kontaktverfolgung wird an verschiedenen Stellen im Raum ein QR"=Code für die
Corona"=Warn"=App (CWA) angebracht. Beim Betreten des Raumes muss sich die betreffende
Person über die CWA einchecken.

Sollte ein Checkin mittels der CWA nicht möglich sein, liegen Zettel aus. Dort
sind die die Angaben zur Kontaktverfolgung einzutragen und in das dafür
vorgesehene Behältnis einzuwerfen.

Der Vorstand oder eine von ihm beauftragte Person leert das Behältnis regelmäßig
und sorgt für eine sichere Vernichtung der Daten am Ende der Aufbewahrungsfrist.

Die Mitglieder werden über das Hygienekonzept informiert. Außerdem wird das Hygienekonzept an der Tür und im Raum ausgehängt.

\section*{Angaben nach §\,5 Abs.\,3 ThürSARS-CoV-2-IfS-MaßnVO vom
  1.\,Juni~2021}

\begin{enumerate}
 \item Verantwortlich gemäß §\,2 Abs.\,2 ThürSARS-CoV-2-MaßnFortentwVO: Vorstand Hackspace Jena e.\,V.
 \item Raumgröße: \SI{60}{\square\metre}
 \item Begehbare Grundstücke unter freiem Himmel: nicht vorhanden
 \item Raumlufttechnische Ausstattung: Vier öffenbare Fenster, eine Außentür.
 \item Maßnahmen zur regelmäßigen Be- und Entlüftung: Beim Kommen und Gehen und regelmäßig während des Aufenthalts wird gelüftet.
 \item Maßnahmen zur weitgehenden Gewährleistung des Mindestabstands: Regelung
   im Hygienekonzept zur Einhaltung des Abstands
 \item Maßnahmen zur angemessenen Beschränkung des Publikumsverkehrs: Der Zugang
   ist nur wenigen Personen gestattet.
 \item Maßnahmen zur Einhaltung der Infektionsschutzregeln: Es werden Möglichkeiten zur Handhygiene bereitgestellt und die Mitglieder werden direkt und per Aushang auf die Hygieneregeln hingewiesen.
 \item Maßnahmen zur Sicherstellung des spezifischen Schutzes der Arbeitnehmer:
   Nicht anwendbar, da keine Arbeitnehmer
 \item Maßnahmen zur Durchführung von Tests: nicht anwendbar
\end{enumerate}

Das vorliegende Hygienekonzept wurde in der Vorstandssitzung des Hackspace Jena e.\,V. am XXXXX bestätigt.

Es wird per E-Mail an alle Mitglieder versendet und über die
\href{https://kraut.space}{Webseite} bekannt gemacht. Zusätzlich liegt das Konzept im Raum aus und wird an der Tür aufgehängt.

Fragen, Kommentare usw. sollen an
\href{mailto:office@kraut.space}{office@kraut.space} gerichtet werden.

\begin{versionhistory}
  \vhEntry{1.0}{10.\,06.\,2020}{Vorstand}{Initiale Version des Dokuments}
  \vhEntry{1.1}{24.\,06.\,2021}{Vorstand}{Anpassung des Konzepts, Regelung von Öffnungsschritten}
\end{versionhistory}

\end{document}

%%% Local Variables:
%%% mode: latex
%%% TeX-master: t
%%% End:
